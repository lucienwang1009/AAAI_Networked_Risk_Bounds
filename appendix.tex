\documentclass{article}

% if you need to pass options to natbib, use, e.g.:
% \PassOptionsToPackage{numbers, compress}{natbib}
% before loading nips_2017
%
% to avoid loading the natbib package, add option nonatbib:
% \usepackage[nonatbib]{nips_2017}

\usepackage{nips_2017}

% to compile a camera-ready version, add the [final] option, e.g.:
% \usepackage[final]{nips_2017}

\usepackage[utf8]{inputenc} % allow utf-8 input
\usepackage[T1]{fontenc}    % use 8-bit T1 fonts
\usepackage{hyperref}       % hyperlinks
\usepackage{url}            % simple URL typesetting
\usepackage{booktabs}       % professional-quality tables
\usepackage{amsfonts}       % blackboard math symbols
\usepackage{nicefrac}       % compact symbols for 1/2, etc.
\usepackage{microtype}      % microtypography

% For using \ifthenelse
\usepackage{ifthen}

\def\GenerateShortVersion{}
\ifthenelse{\isundefined{\GenerateShortVersion}}
{
\def\LongVersion{}
\def\LongVersionEnd{}
\long\def\ShortVersion#1\ShortVersionEnd{}
}
{
\def\ShortVersion{}
\def\ShortVersionEnd{}
\long\def\LongVersion#1\LongVersionEnd{}
}

\def\GenerateSingleColumn{}
\ifthenelse{\isundefined{\GenerateSingleColumn}}
{
\def\DoubleColumn{}
\def\DoubleColumnEnd{}
\long\def\SingleColumn#1\SingleColumnEnd{}
}
{
\def\SingleColumn{}
\def\SingleColumnEnd{}
\long\def\DoubleColumn#1\DoubleColumnEnd{}
}

% Set the typeface to Times Roman
% \usepackage{times}

%% Useful packages
\usepackage{amsmath}
\usepackage{amsthm}
\usepackage{amsfonts}
\usepackage{graphicx}
\usepackage{epstopdf}
\usepackage{color,colortbl}
\definecolor{Gray}{gray}{0.7}
\usepackage{dsfont}
\usepackage{breqn}
\usepackage{mathrsfs}
\usepackage{breakcites}
\usepackage{etoolbox}
\usepackage{subfigure}
\usepackage{bm}
\usepackage{multirow}
\usepackage{array,booktabs}
\newcolumntype{P}[1]{>{\centering\arraybackslash}p{#1}}
\newcolumntype{M}[1]{>{\centering\arraybackslash}m{#1}}
\patchcmd{\thebibliography}{\section*{\refname}}{}{}{}
% \allowdisplaybreaks

\newtheorem{theorem}{Theorem}
\newtheorem{lemma}{Lemma}
\newtheorem{proposition}{Proposition}
\newtheorem{corollary}{Corollary}
\newtheorem{definition}{Definition}
\newtheorem{example}{Example}
\newtheorem*{example*}{Example}
\newtheorem{remark}{Remark}
\newtheorem*{remark*}{Remark}
\newtheorem{assumption}{Assumption}
\newtheorem*{assumption*}{Assumption}
\newcommand{\E}{\mathbb{E}}
\newcommand{\Pro}{P}
\newcommand{\Var}{\text{Var}}
\newcommand{\indicator}{\mathds 1}
\newcommand{\overbar}[1]{\mkern 1.5mu\overline{\mkern-1.5mu#1\mkern-1.5mu}\mkern 1.5mu}
\newcommand*\cen[1]{\overbar{#1}}
\newcommand{\weight}{\mathbf{w}}
\newcommand{\probdistri}{\mathbf{p}}
\newcommand{\verticeweight}{\bar{\mathbf{w}}}
\newcommand{\real}{\mathbb{R}}
\newcommand{\xspace}{\mathcal{X}}
\newcommand{\yspace}{\mathcal{Y}}
\newcommand{\zspace}{\mathcal{Z}}
\newcommand{\bayeserror}{L^*}
\newcommand{\empiricalrisk}[1]{L_{#1}}
\newcommand{\distribution}{P}
\newcommand{\risk}{L}
\newcommand{\lossf}{\ell}
\newcommand{\relossf}{\bar{\ell}}
\newcommand{\cenprocess}[1]{\cen{\gamma_{#1}}}
\newcommand{\rademacher}{\sigma}
\newcommand{\edge}[1]{\{#1\}}
\newcommand{\pair}[1]{(#1)}
\newcommand{\lebesgue}{\mathbb{L}}
\newcommand{\bmlambda}{\bm{\lambda}}
% a better name
\newcommand{\problemabbr}{\textnormal{C}\textsc{tnd}}
% Symmetric relational classification
\newcommand{\normo}[1]{\|#1\|_1}
\newcommand{\fcoloring}{\chi^*}
\newcommand{\complexbound}{\beta}
\newcommand{\citec}[2]{\citeauthor{#1}\ (#2 \citeyear{#1})}
\newcommand{\set}[1]{\{#1\}}
\newcommand{\trainingset}{S}
\newcommand{\rademachercomplexity}{\mathfrak{R}}

\title{On the ERM Principle with Networked Data\\Appendix}

\begin{document}
\maketitle

% \section*{appendix} % (fold)
\appendix
\label{sec:appendix}

This appendix is organized as follows. Section \ref{sec:fractional_coloring_approach} provides new risk bounds for weighted ERM following the Janson's decomposition and show that it cannot improve prior results.
Section \ref{sub:weighted_risk_bounds} establishes the tight risk bounds for weighted ERM on i.i.d. examples under the "low-noise" condition and proves upper bounds involved in covering number for weighted empirical processes. 
Section \ref{sec:a_moment_inequality_for_weighted_u_processes} presents the classical symmetrization and randomization tricks and decoupling inequality for degenerated weighted $U$-processes and the degenerated part $\widetilde{U}_\weight{}(r)$.
Then, some useful inequalities including the moment inequality are proved for weighted Rademacher chaos.
Section \ref{sec:technical_proofs} and \ref{sec:proofs_of_the_appendix} provide technical proofs for the main track and the appendix.
Section \ref{sub:metric_entropy_inequality} introduces the Khinchine inequality and gives the metric entropy inequality for Rademacher chaos.

\section{Janson's decomposition} % (fold)
\label{sec:fractional_coloring_approach}

There are some literature that uses Janson's decomposition derived from \cite{janson2004large} to study the problem learning from networked examples \citep{Usunier2005,Biau2006,ralaivola2009chromatic,DBLP:conf/icml/RalaivolaA15}.
They usually model the networked data with the line graph of $G$.
\begin{definition}[line graph]
  \label{de:line graph}
  Let $G=(V,E)$ be a data graph we consider in the main track. We define the line graph of $G$ as a graph $D_G = (D_V, D_E)$, in which $D_V = E$ and $\edge{i,j}\in D_E$ if and only if $e_i\cap e_j\neq\emptyset$ in $G$.
\end{definition}
This framework differs from our setting and detains less information from the data graph, as argued in Section~1.1. 
One can annalyze this framework by the fractional coloring and the Janson's decomposition.
% Our graph can be easily transformed to the line graph and we denote by $D_G=(D_V, D_E)$ the line graph of the data graph $G$.

\begin{definition}[fractional coloring]
  Let $D_G=(D_V,D_E)$ be a graph. $\mathcal{C}=\set{(\mathcal{C}_j,q_j)}_{j\in (1,\dots,J)}$, for some positive integer $J$, with $\mathcal{C}_j\subseteq D_V$ and $q_j\in [0,1]$ is an fractional coloring of $D_G$, if
  \begin{itemize}
    \item $\forall j$, $\mathcal{C}_j$ is an independent set, i.e., there is no connection between vertices in $C_j$.
    \item it is an exact cover of $G$: $\forall v \in D_V$, $\sum_{j: v\in \mathcal{C}_j} q_j = 1$.
  \end{itemize}
\end{definition}
The weight $W(\mathcal{C})$ of $\mathcal{C}$ is given by $W(\mathcal{C})=\sum_{j=1}^J q_j$ and the minimum weight $\fcoloring{}(D_G)=\min_{\mathcal{C}} W(\mathcal{C})$ over the set of all fractional colorings is the \emph{fractional chromatic number} of $D_G$.
By the Janson's decomposition, one splits all examples into several independent sets according to a fraction coloring of $D_G$ and then analyze each set using the standard method for i.i.d.\ examples.

In particular, Theorem~\ref{th:fractional_mcdiarmid} \citep[see][Theorem~$2$]{Usunier2005} provides a variation of McDiarmid's theorem using the Janson's decomposition.
For simplicity, let $\trainingset=\set{z_i}_{i=1}^{m}$ be the training examples drawn from $\zspace^m$ and $S_{\mathcal{C}_j}$ be the examples included in $\mathcal{C}_j$.
Also, let $k_j^i$ be the index of the $i$-th example of $\mathcal{C}_j$ in the training set $S$ and denote by $\weight{}=\set{w_i}_{i=1}^m$ the weights on $S$.

\begin{theorem}
   \label{th:fractional_mcdiarmid}
   Using the notations defined above, let $\mathcal{C} = \set{(\mathcal{C}_j, q_j)}_{j=1}^J$ be a fractional coloring of $D_G$. Let $f:\zspace^m \to \real{}$ such that
   \begin{itemize}
     \item there exist $J$ functions $\zspace^{|\mathcal{C}_j|} \to \real{}$ which satisfy $\forall S\in \zspace^m$, $f(S)=\sum_{j=1}^J q_j f_j(S_{\mathcal{C}_j})$.
     \item there exist $c_i,\dots,c_m\in \real{}_+$ such that $\forall j,\forall S_{\mathcal{C}_j}, S_{\mathcal{C}_j}^k$ such that $S_{\mathcal{C}_j}$ and $S_{\mathcal{C}_j}^k$ differ only in the $k$-th example, $|f_j(S_{\mathcal{C}_j}) - f_j(S_{\mathcal{C}_j}^k)|\le c_{k_j^i}$.
   \end{itemize}
     Then, we have
     \[\Pro[f(S)-\E[f(S)]\ge \epsilon] \le \exp\left(-\frac{2\epsilon^2}{\fcoloring{}(D_G)\sum_{i=1}^mc_i^2}\right).\]
 \end{theorem}

\subsection{Weighted ERM} % (fold)
\label{sub:weighted_erm}
% \subsection{Weithed ERM}

Using the above theorem, we show that the risk bounds for weighted ERM derived from the Janson's decomposition cannot improve the results of \cite{Usunier2005}, as shown in the following theorem.

\begin{theorem}
\label{th:fractional_coloring_weighted}
  Consider a minimizer $r_\weight{}$ of the weighted empirical risk $\empiricalrisk{\weight{}}$ over a class $R$. For all $\delta\in (0,1]$, with probability at least $1-\delta$, we have
  \[\risk{}(r)-\empiricalrisk{\weight{}}(r)\le \rademachercomplexity_{\weight{}}^*(R,S)+\sqrt{\fcoloring(D_G)\log(1/\delta)}\frac{\|\weight{}\|_2}{\normo{\weight{}}}\]
  where
  \[\rademachercomplexity_{\weight{}}^*(R, S) = \frac{2}{N}\E_\rademacher{}\left[\sum_{j=1}^J q_j\sup_{r\in R} \sum_{i\in \mathcal{C}_j}w_{k_j^i}\rademacher{}_i r(z_{k_j^i})\right].\]
  is the weighted empirical fractional Rademacher complexity of $R$ with respect to $D_G$.
\end{theorem}

In this setting, although the weights $\weight{}$ are no limit to the fractional matching, the risk bounds cannot improve the results of \cite{Usunier2005}, which is of the order $\sqrt{\fcoloring{}(D_G)/m}$, as $\|\weight\|_2/\normo{\weight{}}\le 1/\sqrt{m}$.

\subsection{``Low-noise'' condition} % (fold)
\label{sub:low_noise_condition}

Considering the complete graph with equal weighting scheme, by the Janson's decomposition, the empirical risk $\empiricalrisk{m}(L)$ can be represented as an average of sums of i.i.d.\ r.v.'s
\begin{equation}
  \label{eq:fractional_coloring_ERM}
  \frac{1}{m!}\sum_{\bm{\pi}\in\mathcal{G}_n}\frac{1}{\lfloor m/2\rfloor}\sum_{i=1}^{\lfloor m/2 \rfloor} \ell(r, (X_{\pi(i)}, X_{\pi(i)+\lfloor m/2\rfloor}, Y_{\pi(i), \pi(i)+\lfloor m/2\rfloor}))
\end{equation}
where the sum is taken over all permutations of $\mathcal{G}_m$, the symmetric group of order $m$, and $\lfloor u \rfloor$ denotes the integer part of any $u\in \real{}$.
From \cite{Biau2006}, the bounds for excess risk of \eqref{eq:fractional_coloring_ERM} are of the order $O(1/\sqrt{n})$.
Moreover, by the result in \cite{DBLP:conf/icml/RalaivolaA15}, tighter risk bounds may be obtained under the following ``low-noise'' condition \citep[see][]{tsybakov2004optimal,Massart2006}.
\begin{assumption}
  There exists $C>0$ and $\theta\in [0,1]$ such that for all $\epsilon>0$,
  \[\Pro\left[|\eta(X_1,X_2)-\frac{1}{2}|\le \epsilon\right]\le C\epsilon^{\theta/(1-\theta)}.\]
\end{assumption}
The risk bounds of the order $O(\log(n)/n)$ may be achieved if $\theta=1$ \citep[\textit{cf.}][]{Massart2006}, which however is very restrictive.
We can use the example in \cite{papa2016graph} to show this.
Let $N$ be a positive integer. For each vetex $i \in \set{1,\dots,n}$, we observe $X_i = (X_i^1, X_i^2)$ where $X_i^1$ and $X_i^2$ are two distinct elements drawn from $\set{1,\dots,N}$.
This may, for instance correspond to the two preferred items of a user $i$ among a list of $N$ items.
Consider now the case that two nodes are likely to be connected if they share common preference, e.g., $Y_{i,j} \sim Ber(\#(X_i\cap X_j)/2)$. One can easily check that $\Pro[|\eta(X_1,X_2)-1/2|=0]>0$, so tight risk bounds cannot be obtained for minimizers of \eqref{eq:fractional_coloring_ERM}.

\section{Tight bounds for weighted ERM on i.i.d.\ examples} % (fold)
\label{sub:weighted_risk_bounds}

In section 3, the excess risk is split into two types of processes: the weighted empirical process of i.i.d.\ examples and two degenerated processes.
In this section, we prove general risk bounds for the weighted empirical process of i.i.d.\ examples. 
The main idea is similar to \cite{Massart2006} that tighter bounds for the excess risk can be obtained if the variance of the excess risk is controlled by its expected value.

\subsection{Bennett concentration inequality} % (fold)
\label{subsub:bennett_type_inequality}

% subsection bennett_type_inequality (end)

First, we prove a concentration inequality for the supremum of weighted empirical processes derived from \cite{Bousquet2002a}.

\begin{theorem}
    \label{th:weighted_bennett_inequality}
    Assume the $(X_1,\dots,X_i)$ are i.i.d.\ random variable according to $P$. Let $\mathcal F$ be a countable set of functions from $\mathcal X$ to $\mathbb R$ and assume that all functions $f$ in $\mathcal F$ are $P$-measurable and square-integrable. If $\sup_{f\in\mathcal F} |f| \le b$, we denote
    \[Z = \sup_{f\in \mathcal F} \frac{1}{\normo{\weight{}}}\sum_{i=1}^n w_i (f(X_i)-\E[f(X_i)]).\]
    which $\{w_i\}_{i=1}^n$ are bounded weights such that $0\le w_i\le 1$ for $i=1,\dots,n$ and $\normo{\weight{}}>0$.
    Let $\sigma$ be a positive real number such that $\sigma^2\ge \sup_{f\in\mathcal F} \Var[f(X)]$ almost surely, then for all $x\ge 0$, we have
    \begin{equation}
        \label{eq:weighted_bennett_inequality}
            \Pro\left[Z-\E[Z] \ge \sqrt{\frac{2(\frac{\|\weight{}\|_2^2}{\normo{\weight{}}}\sigma^2+4b\E[Z])x}{\normo{\weight{}}}} + \frac{2bx}{3\normo{\weight{}}}\right] \le e^{-x}.
    \end{equation}
    % where $\normo{\weight{}} = \sum_{i=1}^n w_i$.
\end{theorem}

It is a variant of Theorem 2.3 of \cite{Bousquet2002a} by just applying Theorem 2.1 of \cite{Bousquet2002a} with the weighted empirical process.
Then, by Theorem \ref{th:weighted_bennett_inequality}, we can generalize the results of \cite{Massart2006} to weighted ERM. We start by describing the probabilistic framework adapts to our problem.

\subsection{General upper bounds} % (fold)
\label{sub:i_i_d_}

% subsection i_i_d_ (end)

Suppose that one observes independent variable $\xi_1,\dots,\xi_n$ taking their values in some measurable space $\mathcal Z$ with commom distribution $P$. 
For every $i$, the variable $\xi_i=(X_i,Y_i)$ is a copy of a pair of random variables $(X,Y)$ where $X$ take its values in measurable space $\xspace$. 
Think of $\mathcal{R}$ as being the set of all measurable functions from $\xspace$ to $\{0,1\}$. 
Then we consider some \emph{loss} function
\begin{equation}
    \label{eq:iid_loss}
    \gamma: \mathcal{R} \times \mathcal{Z} \to [0,1].
\end{equation}
Basically one can consider some set $\mathcal{R}$, which is known to contain the \emph{Bayes classifier} $r^*$ that achieves the best (smallest) \emph{expected loss} $P[\gamma(r,\cdot)]$ when $r$ varies in $\mathcal{R}$. The \emph{relative expected loss $\relossf$} is defined by
\begin{equation}
    \label{eq:iid_relative_expected_loss}
    \relossf(r^*,r) = \Pro[\gamma(r,\cdot)-\gamma(r^*,\cdot)], \forall r\in\mathcal{R}
\end{equation}
Since the empirical process on i.i.d.\ examples split from the excess risk is with non-negative weights on all examples, we define the \emph{weighted loss} as
\begin{equation}
    \label{eq:weighted_erm}
    \gamma_\weight(r) = \frac{1}{\normo{\weight{}}}\sum_{i=1}^n w_i\gamma(r,\xi_i)
\end{equation}
where $0\le w_i\le 1$ for all $i=1,\dots,n$ and $\normo{\weight{}}=\sum_{i=1}^n w_i> 0$. Weighted ERM approach aims to find a minimizer of the weighted empirical loss $\hat{r}$ in the hypothesis set $R\subset \mathcal{R}$ to approximate $r^*$.

We introduce the \emph{weighted centered empirical process} $\cenprocess{\weight}$ defined by
\begin{equation}
    \label{eq:centered_empirical_process}
    \cenprocess{\weight}(r) = \gamma_\weight(r)-\Pro[\gamma(r,\cdot)].
\end{equation}
In addition to the relative expected loss function $\relossf$, we shall need another way to measure the closeness between the elements of $R$.
\LongVersion
which is directly connected to the variance of the increments of $\cenprocess{\weight}$ and therefore will play an important role in the analysis of the fluctuations of $\cenprocess{\weight}$.
\LongVersionEnd
Let $d$ be some pseudo-distance on $\mathcal{R}\times\mathcal{R}$ such that
\begin{equation}
    \label{eq:variancD_Eistance_inequality}
    \Var[\gamma(r,\cdot)-\gamma(r^*,\cdot)]\le d^2(r,r^*), \forall r\in\mathcal{R}.
\end{equation}

A tighter risk bound for weighted ERM is derived from Theorem \ref{th:weighted_bennett_inequality} which combines two different moduli of uniform continuity: the stochastic modulus of uniform continuity of $\cenprocess{\weight}$ over $R$ with respect to $d$ and the modulus of uniform continuity of $d$ with respect to $\relossf$.

Next, we need to specify some mild regularity conditions functions that we shall assume to be verified by the moduli of continuity involved in the following result.

\begin{definition}
\label{con:nondecreasing_condition}
    We denote by $D$ the class of nondecreasing and continuous functions $\psi$ from $\mathbb R_+$ to $\mathbb R_+$ such that $x\to \psi(x)/x$ is nonincreasing on $(0,+\infty)$ and $\psi(1)\ge 1$.
\end{definition}
% In particular, for the applications that we shall study below, an example of special interest is $\phi(x)=Ax^\alpha$, where $\alpha\in (0,1]$ and $A\ge 1$.

In order to avoid measurability problems, we need to consider some separability condition on $R$. The following one will be convenient.

\begin{assumption}
\label{con:separability_condition}
    There exists some countable subset $R^\prime$ of $R$ such that, for every $r\in R$, there exists some sequence $\{r_k\}$ of elements of $R^\prime$ such that, for every $\xi \in \mathcal Z$, $\gamma(r_k, \xi)$ tends to $\gamma(r,\xi)$ as $k$ tends to infinity.
\end{assumption}

The upper bound for the relative expected loss of any empirical risk minimizer on some given model $R$ will depend on the bias term $\relossf(r^*,R)=\inf_{r\in R}\relossf(r^*,r)$ and the fluctuations of the empirical process $\cenprocess{\weight{}}$ on $R$. As a matter of fact, we shall consider some slightly more general estimators. Namely, given some nonnegative number $\rho$, we consider some $\rho$-empirical risk minimizer, that is, any estimator $r$ taking its values in $R$ such that $\gamma_\weight(\hat{r})\le \rho+\inf_{r\in R}\gamma_\weight(r)$.

\begin{theorem}[risk bound for weighted ERM]
    \label{th:weighted_erm_risk_bounds}
    Let $\gamma$ be a loss function such $r^*$ minimizes $\Pro[\gamma(r,\cdot)]$ when $r$ varies in $\mathcal{R}$. Let $\phi$ and $\psi$ belong to the class of functions $D$ defined above and let $R$ be a subset of $\mathcal R$ satisfying the sparability Assumption \ref{con:separability_condition}. Assume that, on the one hand,
    \begin{equation}
        \label{eq:weighted_erm_risk_bound_con1}
        d(r^*,r)\le \frac{\sqrt{\normo{\weight{}}}}{\|\weight{}\|_2}\psi(\sqrt{\relossf(r^*,r)}), \forall r\in \mathcal R,
    \end{equation}
    and that, on the other hand, one has, for every $r\in R^\prime$,
    \begin{equation}
        \label{eq:weighted_erm_risk_bound_con2}
        \sqrt{\normo{\weight{}}}\E\left[\sup_{r'\in R^\prime,\frac{\|\weight{}\|_2}{\sqrt{\normo{\weight{}}}}d(r',r)\le \sigma}[\cenprocess{\weight}(r')-\cenprocess{\weight}(r)\right]\le \phi(\sigma)
    \end{equation}
    for every positive $\sigma$ such that $\phi(\sigma)\le \sqrt{\normo{\weight{}}}\sigma^2$, where $R^\prime$ is given by Assumption \ref{con:separability_condition}. Let $\epsilon_*$ be the unique positive solution of the equation
    \begin{equation}
        \label{eq:risk_bound_con3}
        \sqrt{\normo{\weight{}}}\epsilon_*^2=\phi(\psi(\epsilon_*)).
    \end{equation}
    Then there exists an absolute constant $K$ such that, for every $y\ge 1$, the following inequality holds:
    \begin{equation}
        \label{eq:risk_bound_probability}
        \Pro\left[\relossf(r^*,\hat{r})>2\rho+2\relossf(r^*,R)+K y\epsilon_*^2\right]\le e^{-y}.
    \end{equation}
\end{theorem}

% something discuss the theorem above.

\subsection{Maximal inequality for weighted empirical processes} % (fold)
\label{sub:maximal_inequality_for_weighted_processes}

% subsection maximal_inequality_for_weighted_processes (end)

Next, we present the maximal inequality involved in covering number for weighted empirical processes. Let us fix some notation. We consider i.i.d.\ random variables $\xi_1,\dots,\xi_n$ with values in some measurable space $\mathcal{Z}$ and common distribution $P$. For any $P$-integrable function $f$ on $\mathcal{Z}$, we define $P_\weight(f)=\frac{1}{\normo{\weight{}}}w_i\sum_{i=1}^nf(\xi_i)$ and $v_\weight(f)=P_\weight(f)-P(f)$ where $0\le w_i\le 1$ for all $i=1,\dots,n$ and $\normo{\weight{}}=\sum_{i=1}^n w_i> 0$. Given a collection $\mathcal{F}$ of $P$-integrable functions $f$, our purpose is to control the expectation of $\sup_{f\in\mathcal{F}}v_\weight(f)$ or $\sup_{f\in\mathcal{F}}-v_\weight(f)$.

\begin{lemma}
\label{le:maximal_entropy_inequality}
    Let $\mathcal{F}$ be a countable collection of measurable functions such that $f\in [0,1]$ for every $f\in\mathcal{F}$, and let $f_0$ be a measurable function such that $f_0\in[0,1]$. Let $\sigma$ be a positive number such that $\Pro[|f-f_0|]\le \sigma^2$ for every $f\in\mathcal{F}$. Then, setting
    \[\varphi(\sigma)=\int_0^{\sigma} (\log N_\infty(\mathcal{F},\epsilon^2))^{1/2}d\epsilon,\]
    the following inequality is available:
    \DoubleColumn
    \begin{align*}
        \sqrt{\normo{\weight{}}}\max(\E[\sup_{f\in\mathcal{F}}v_\weight(f_0-f)],\E[\sup_{f\in\mathcal{F}} &v_\weight(f-f_0)])\\
        &\le 12\varphi(\sigma).
    \end{align*}
    \DoubleColumnEnd
    \SingleColumn
    \[\sqrt{\normo{\weight{}}}\max\left(\E[\sup_{f\in\mathcal{F}}v_\weight(f_0-f)],\E[\sup_{f\in\mathcal{F}} v_\weight(f-f_0)]\right)\le 12\varphi(\sigma).\]
    \SingleColumnEnd
    provided that $4\varphi(\sigma)\le \sigma^2\sqrt{\normo{\weight{}}}$.
\end{lemma}

% subsection risk_bounds_of_weighted_erm_of_i_i_d_samples (end)

\section{Inequalities for $U_\weight{}(r)$ and $\widetilde{U}_\weight{}(r)$} % (fold)
\label{sec:a_moment_inequality_for_weighted_u_processes}

In this section, we first show the classical symmetrization and randomization tricks for the degenerated weighted $U$-statistics $U_\weight{}(r)$ and the degenerated part $\widetilde{U}_\weight{}(r)$.
Then we establish general exponential inequalities for weighted Rademacher chaos.
This result is generalized from \cite{clemenccon2008ranking} based on moment inequalities obtained for empirical processes and Rademacher chaos in \cite{Boucheron2005}.
With this moment inequality, we prove the inequality for weighted Rademacher chaos, which involves the $\lebesgue{}_\infty$ covering number of the hypothesis set.

\begin{lemma}[decoupling and undecoupling]
    \label{le:weighted_u_statistics_decoupling}
    Let $(X_i^\prime)_{i=1}^n$ be an independent copy of the sequence $(X_i)_{i=1}^n$. Then, for all $q\ge 1$, we have:
    \DoubleColumn
    \begin{equation}
        \begin{aligned}
        \label{eq:weighted_u_statistics_decoupling}
            \E[\sup_{f_{i,j}\in\mathcal{F}}&|\sum_{\pair{i,j}\in E}w_{i,j} f_{i,j}(X_i,X_j)|^q] \\
            &\le 4^q\E[\sup_{f_{i,j}\in\mathcal{F}}|\sum_{\pair{i,j}\in E} w_{i,j}f_{i,j}(X_i,X_j^\prime)|^q]
        \end{aligned}
    \end{equation}
    \DoubleColumnEnd
    \SingleColumn
    \begin{equation}
        \begin{aligned}
        \label{eq:weighted_u_statistics_decoupling}
            \E[\sup_{f_{i,j}\in\mathcal{F}}&|\sum_{\pair{i,j}\in E}w_{i,j} f_{i,j}(X_i,X_j)|^q] \le 4^q\E[\sup_{f_{i,j}\in\mathcal{F}}|\sum_{\pair{i,j}\in E} w_{i,j}f_{i,j}(X_i,X_j^\prime)|^q]
        \end{aligned}
    \end{equation}
    \SingleColumnEnd
    If the functions $f_{i,j}$ are symmetric in the sense that for all $X_i,X_j$,
    \[f_{i,j}(X_i,X_j)=f_{j,i}(X_j,X_i)\]
    and $(w_{i,j})_{\pair{i,j}\in E}$ is symmetric, then the inequality can be reversed, that is,
    \DoubleColumn
    \begin{equation}
        \begin{aligned}
        \label{eq:weighted_u_statistic_coupling}
        \E[\sup_{f_{i,j}\in\mathcal{F}}&|\sum_{\pair{i,j}\in\E}w_{i,j} f_{i,j}(X_i,X_j^\prime)|^q]\\
        &\le 4^q \E[\sup_{f_{i,j}\in\mathcal{F}}|\sum_{\pair{i,j}\in E}w_{i,j} f_{i,j}(X_i,X_j)|^q]
        \end{aligned}
    \end{equation}
    \DoubleColumnEnd
    \SingleColumn
    \begin{equation}
        \begin{aligned}
        \label{eq:weighted_u_statistic_coupling}
        \E[\sup_{f_{i,j}\in\mathcal{F}}&|\sum_{\pair{i,j}\in\E}w_{i,j} f_{i,j}(X_i,X_j^\prime)|^q] \le 4^q \E[\sup_{f_{i,j}\in\mathcal{F}}|\sum_{\pair{i,j}\in E}w_{i,j} f_{i,j}(X_i,X_j)|^q]
        \end{aligned}
    \end{equation}
    \SingleColumnEnd
\end{lemma}
\begin{lemma}[randomization]
    \label{le:weighted_u_statistics_randomization}
    Let $(\sigma_i)_{i=1}^n$ and $(\sigma_i^\prime)_{i=1}^n$ be two independent sequences of i.i.d.\ Radermacher variables, independent from the $(X_i,X_i^\prime)$'s. If $f$ is degenerated, we have for all $q\ge 1$,
    \DoubleColumn
    \begin{align*}
        \E[\sup_{f\in\mathcal{F}}&|\sum_{\pair{i,j}\in E} w_{i,j}f(X_i,X_j^\prime)|^q]\\
        &\le 4^q\E[\sup_{f\in\mathcal{F}}|\sum_{\pair{i,j}\in E}\sigma_i\sigma_j^\prime w_{i,j}f(X_i, X_j^\prime)|^q]
    \end{align*}
    \DoubleColumnEnd
    \SingleColumn
    \begin{align*}
        \E[\sup_{f\in\mathcal{F}}&|\sum_{\pair{i,j}\in E} w_{i,j}f(X_i,X_j^\prime)|^q]\le 4^q\E[\sup_{f\in\mathcal{F}}|\sum_{\pair{i,j}\in E}\sigma_i\sigma_j^\prime w_{i,j}f(X_i, X_j^\prime)|^q]
    \end{align*}
    \SingleColumnEnd
\end{lemma}
\begin{lemma}
    \label{le:weighted_decoupling_undecoupling}
    Let $(X_i^\prime)_{i=1}^n$ be an independent copy of the sequence $(X_i)_{i=1}^n$. Consider random variables valued in $\{0,1\}$, $(\tilde{Y}_{i,j})_{\pair{i,j}\in E}$, conditionally independent given the $X_i^\prime$'s and the $X_i$'s and such that $\Pro[\tilde{Y}_{i,j}=1\mid X_i,X_j^\prime]=\eta(X_i, X_j^\prime)$. We have for all $q\ge 1$,
    \DoubleColumn
    \begin{equation}
        \begin{aligned}
        \label{eq:weighted_u_statistics_decoupling}
            \E[\sup_{r\in R}&|\sum_{\pair{i,j}\in E}w_{i,j} \tilde{h}_r(X_i,X_j, Y_{i,j})|^q] \\
            &\le 4^q\E[\sup_{r\in R}|\sum_{\pair{i,j}\in E} w_{i,j}\tilde{h}_r(X_i,X_j^\prime, \tilde{Y}_{i,j})|^q]
        \end{aligned}
    \end{equation}
    \DoubleColumnEnd
    \SingleColumn
    \begin{equation}
        \begin{aligned}
        \label{eq:weighted_u_statistics_decoupling}
            \E[\sup_{r\in R}&|\sum_{\pair{i,j}\in E}w_{i,j} \tilde{h}_r(X_i,X_j, Y_{i,j})|^q] \le 4^q\E[\sup_{r\in R}|\sum_{\pair{i,j}\in E} w_{i,j}\tilde{h}_r(X_i,X_j^\prime, \tilde{Y}_{i,j})|^q]
        \end{aligned}
    \end{equation}
    \SingleColumnEnd
    and the inequality can be reversed,
    \DoubleColumn
    \begin{equation}
        \begin{aligned}
        \label{eq:weighted_u_statistic_coupling}
        \E[\sup_{r\in R}&|\sum_{\pair{i,j}\in\E}w_{i,j} \tilde{h}_r(X_i,X_j^\prime, \tilde{Y}_{i,j})|^q]\\
        &\le 4^q \E[\sup_{r\in R}|\sum_{\pair{i,j}\in E}w_{i,j} \tilde{h}_r(X_i,X_j,Y_{i,j})|^q]
        \end{aligned}
    \end{equation}
    \DoubleColumnEnd
    \SingleColumn
    \begin{equation}
        \begin{aligned}
        \label{eq:weighted_u_statistic_coupling}
        \E[\sup_{r\in R}&|\sum_{\pair{i,j}\in\E}w_{i,j} \tilde{h}_r(X_i,X_j^\prime, \tilde{Y}_{i,j})|^q] \le 4^q \E[\sup_{r\in R}|\sum_{\pair{i,j}\in E}w_{i,j} \tilde{h}_r(X_i,X_j,Y_{i,j})|^q]
        \end{aligned}
    \end{equation}
    \SingleColumnEnd
\end{lemma}

\begin{lemma}
    \label{le:weighted_randomization}
    Let $(\sigma_i)_{i=1}^n$ and $(\sigma_i^\prime)_{i=1}^n$ be two independent sequences of i.i.d.\ Radermacher variables, independent from the $(X_i,X_i^\prime, Y_{i,j}, \tilde{Y}_{i,j})$'s. Then, we have for all $q\ge 1$,
    \DoubleColumn
    \begin{align*}
        \E[\sup_{r\in R}&|\sum_{\pair{i,j}\in E} w_{i,j}\tilde{h}_r(X_i,X_j^\prime, \tilde{Y}_{i,j})|^q]\\
        &\le 4^q\E[\sup_{r\in R}|\sum_{\pair{i,j}\in E}\sigma_i\sigma_j^\prime w_{i,j}\tilde{h}_r(X_i, X_j^\prime,\tilde{Y}_{i,j})|^q]
    \end{align*}
    \DoubleColumnEnd
    \SingleColumn
    \begin{align*}
        \E[\sup_{r\in R}&|\sum_{\pair{i,j}\in E} w_{i,j}\tilde{h}_r(X_i,X_j^\prime, \tilde{Y}_{i,j})|^q] \le 4^q\E[\sup_{r\in R}|\sum_{\pair{i,j}\in E}\sigma_i\sigma_j^\prime w_{i,j}\tilde{h}_r(X_i, X_j^\prime,\tilde{Y}_{i,j})|^q]
    \end{align*}
    \SingleColumnEnd
\end{lemma}


Lemma \ref{le:weighted_u_statistics_decoupling} and \ref{le:weighted_u_statistics_randomization} are applications of Theorem 3.1.1 and Theorem 3.5.3 of \cite{de2012decoupling} respectively, providing decoupling and randomization inequalities for degenerated weighted $U$-statistics of order $2$. Lemma \ref{le:weighted_decoupling_undecoupling} and \ref{le:weighted_randomization} are the variants of Lemma \ref{le:weighted_u_statistics_decoupling} and \ref{le:weighted_u_statistics_randomization} respectively, which is suitable for the degenerated part $\widetilde{U}_\weight{}(r)$.

\begin{theorem}[moment inequality]
    \label{th:moment_inequality_of_weighted_u_statistics}
    Let $X, X_1,\dots,X_n$ be i.i.d.\ random variables and let $\mathcal F$ be a class of kernels. Consider a weighted Rademacher chaos $Z_\rademacher{}$ of order $2$ on the graph $G=(V,E)$ indexed by $\mathcal F$,
    \[Z = \sup_{f\in\mathcal F}|\sum_{\pair{i,j}\in E}w_{i,j}f(X_i,X_j)|\]
    where $\E[f(X,x)]=0$ for all $x\in\xspace{},f\in\mathcal{F}$. Assume also for all $x,x'\in\xspace$, $f(x,x')=f(x',x)$ (symmetric) and $\sup_{f\in\mathcal F}\|f\|_\infty=F$. Let $(\rademacher{}_i)_{i=1}^n$ be i.i.d.\ Rademacher random variables and introduce the random variables
    \[Z_\rademacher = \sup_{f\in\mathcal F}|\sum_{\pair{i,j}\in E}w_{i,j}\rademacher_i\rademacher_jf(X_i,X_j)|\]
    \[U_\rademacher = \sup_{f\in\mathcal F}\sup_{\alpha:\|\alpha\|_2\le 1}\sum_{\pair{i,j}\in\E}w_{i,j}\rademacher_i\alpha_jf(X_i,X_j)\]
    \[M=\sup_{f\in\mathcal F,k=1,\dots,n}|\sum_{i:(i,k)\in E}w_{i,k}\rademacher_if(X_i,X_k)|\]
    Then exists a universal constant $C$ such that for all $n$ and $t>0$,
    \DoubleColumn
    \begin{align*}
        \Pro[Z&\ge C\E[Z_\rademacher]+t]\\
        &\le \exp\Bigg(-\frac{1}{C}\min\Big((\frac{t}{\E[U_\rademacher]})^2,\frac{t}{\E[M]+F\|\weight{}\|_2},\\
        &\qquad(\frac{t}{\|\weight{}\|_{\max}F})^{2/3},\sqrt{\frac{t}{\|\weight{}\|_\infty F}}\Big)\Bigg)
    \end{align*}
    \DoubleColumnEnd
    \SingleColumn
    \begin{align*}
        \Pro[Z&\ge C\E[Z_\rademacher]+t]\\
        &\le \exp\Bigg(-\frac{1}{C}\min\Big((\frac{t}{\E[U_\rademacher]})^2,\frac{t}{\E[M]+F\|\weight{}\|_2},(\frac{t}{\|\weight{}\|_{\max}F})^{2/3},\sqrt{\frac{t}{\|\weight{}\|_\infty F}}\Big)\Bigg)
    \end{align*}
    \SingleColumnEnd
    % there exists a universal constant $C>0$ such that for all $n$ and $q\ge 2$,
    % \DoubleColumn
    % \begin{align*}
    %     (\E[Z^q])^{1/q}&\le C(\E[Z_\rademacher]+q^{1/2}\E[U_\rademacher]+q\E[M]\\
    %     &+\|\weight{}\|_\infty Fq^2+\|\weight{}\|_{\max}Fq^{3/2}+F\|\weight{}\|_2q)
    % \end{align*}
    % \DoubleColumnEnd
    % \SingleColumn
    % \[(\E[Z^q])^{1/q}\le C(\E[Z_\rademacher]+q^{1/2}\E[U_\rademacher]+q\E[M]+\|\weight{}\|_\infty Fq^2+\|\weight{}\|_{\max}Fq^{3/2}+F\|\weight{}\|_2q)\]
    % \SingleColumnEnd
    where $\|\weight{}\|_{\max}=\max_i \sqrt{\sum_{j:\pair{i,j}\in E} w_{i,j}^2}$.
\end{theorem}

If the hypothesis set $\mathcal{F}$ is a subset of $\lebesgue{}_\infty(\xspace^2)$ (upper bounds on the uniform covering number with $\lebesgue{}_\infty$ metric can be calculated, see \cite{cucker2007learning}), we show $\E[Z_\rademacher{}]$, $\E[U_\rademacher{}]$ and $\E[M]$ can be bounded by $N_\infty(\mathcal{F},\epsilon)$ since all these Rademacher random variables satisfy the Khinchine inequality (see Section~\ref{sub:metric_entropy_inequality}). Following the metric entropy inequality for Khinchine-type processes (see Section~\ref{sub:metric_entropy_inequality}), it is easy to get the following Corollary.

\begin{corollary}
    \label{cor:covering_number_inequality}
    With the same setting of Theorem~\ref{th:moment_inequality_of_weighted_u_statistics}, if $\mathcal{F}\subset \lebesgue{}_\infty(\xspace{}^2)$, we have for any $\delta<1/e$,
    \[\Pro[Z\le \kappa]\ge 1-\delta\]
    where 
    \begin{align*}
        \kappa\le&C\Bigg(\|\weight{}\|_2\int_0^{2F}\log N_\infty(\mathcal F, \epsilon)d\epsilon+\max\Big(\|\weight{}\|_2\log(1/\delta)\int_0^{2F}\sqrt{\log N_\infty(\mathcal F,\epsilon)}d\epsilon,\\
        &\qquad\qquad(\log(1/\delta))^{3/2}\|\weight{}\|_{\max}, (\log(1/\delta))^2\|\weight{}\|_\infty\Big)\Bigg).
    \end{align*}
    with a universal constant $C$.
\end{corollary}


% \subsection{Decoupling and randomization} % (fold)
% \label{sub:decoupling_coupling_and_randomization}

% subsection decoupling_coupling_and_randomization (end)

% In this section, we provide decoupling and randomization inequalities for degenerated weighted $U$-statistics and the degenerated part $\widetilde{U}_\weight{}(r)$.
% \begin{proof}
%     This lemma is derived from Theorem 3.5.3 of \citep{Gani}.

%     Re-using the notations used in the proof of Lemma \ref{le:weighted_u_statistics_decoupling}, we further introduce $(X_i^{\prime\prime})_{i=1}^n$, a copy of $(X_i^\prime)_{i=1}^n$, independent from $\mathcal F$, $\mathcal F^\prime$, and denote by $\mathcal F^{\prime\prime}$ its sigma-field. We now use classic randomization techniques and introduce our "ghost" sample:
% \begin{align*}
%     &\E[\sup_f|\sum_{\pair{i,j}\in E} w_{i,j}f(X_i,X_j^\prime)|^q]\\
%     &(f \text{ is degenerated})\\
%     &= \E[\sup_f|\sum_{\pair{i,j}\in E} w_{i,j}(f(X_i,X_j^\prime)-\E_{\mathcal F^{\prime\prime}}[f(X_i,X_j^{\prime\prime})])|^q]\\
%     &(\text{Jansen's Inequality})\\
%     &\le \E[\sup_f|\sum_{\pair{i,j}\in E} w_{i,j}(f(X_i,X_j^\prime)-f(X_i,X_j^{\prime\prime}))|^q]\\
%     &= \E[\sup_f|\sum_{j=1}^n \sum_{\pair{i,j}\in E} w_{i,j}(f(X_i,X_j^\prime)-f(X_i,X_j^{\prime\prime}))|^q]
% \end{align*}
% Let $(\sigma_i)_{i=1}^n$ be independent Rademacher variables, independent of $\mathcal F$, $\mathcal F^\prime$ and $\mathcal F^{\prime\prime}$, then we have:
% \begin{align*}
%     \E[&\sup_f|\sum_{j=1}^n \sum_{\pair{i,j}\in E} w_{i,j}(f(X_i,X_j^\prime)-f(X_i,X_j^{\prime\prime}))|^q\mid \mathcal F]\\
%     &= \E[\sup_f|\sum_{j=1}^n \sigma_j \sum_{\pair{i,j}\in E} w_{i,j}(f(X_i,X_j^\prime)\\
%     &\qquad-f(X_i,X_j^{\prime\prime}))|^q\mid \mathcal F]\\
%     &(\text{Triangle's Inequality and the convexity of }\sup_f|\cdot|^q)\\
%     &\le 2^q\E[\sup_f|\sum_{j=1}^n \sigma_j \sum_{\pair{i,j}\in E} w_{i,j}f(X_i,X_j^\prime)|^q\mid \mathcal F]\\
%     &\qquad+ 2^q\E[\sup_f|\sum_{j=1}^n \sigma_j \sum_{\pair{i,j}\in E} w_{i,j}f(X_i,X_j^{\prime\prime}))|^q\mid \mathcal F]\\
%     &\le 2^q\E[\sup_f|\sum_{j=1}^n\sigma_j\sum_{\pair{i,j}\in E}w_{i,j} f(X_i,X_j^\prime)|^q\mid \mathcal F]
% \end{align*}
% and get
% \begin{align*}
%     \E[\sup_f&|\sum_{\pair{i,j}\in E} w_{i,j}f(X_i,X_j^\prime)|^q] \\
%     &\le 2^q\E[\sup_f|\sum_{j=1}^n\sigma_j\sum_{\pair{i,j}\in E}w_{i,j} f(X_i,X_j^\prime)|^q]
% \end{align*}
% We now repeat the same argument but for the $(X_i)_{i=1}^n$. Let $(X_i^{\prime\prime\prime})_{i=1}^n$ be a copy of $(X_i)_{i=1}^n$, independent of $\mathcal F$, $\mathcal F^\prime$, and denote by $\mathcal F^{\prime\prime\prime}$ its sigma-field. Similarly we have
% \begin{align*}
%     \E[\sup_f&|\sum_{j=1}^n\sigma_j\sum_{\pair{i,j}\in E}w_{i,j} f(X_i,X_j^\prime)|^q]\\
%     &\le \E[\sup_f|\sum_{i=1}^n \sum_{\pair{i,j}\in E} w_{i,j}(\sigma_j f(X_i,X_j^\prime)\\
%     &\qquad-\sigma_j f(X_i^{\prime\prime\prime},X_j^{\prime}))|^q]
% \end{align*}
% Let $(\sigma_i^\prime)_{i=1}^n$ be a copy of $(\sigma_i)_{i=1}^n$, independent of $(\sigma_i)_{i=1}^n$, $\mathcal F$, $\mathcal F^\prime$, $\mathcal F^{\prime\prime\prime}$, we have
% \begin{align*}
%     \E[\sup_f&|\sum_{i=1}^n \sum_{\pair{i,j}\in E} w_{i,j}(\sigma_j f(X_i,X_j^\prime)\\
%     &\qquad-\sigma_j f(X_i^{\prime\prime\prime},X_j^{\prime}))|^q\mid \mathcal F^\prime, (\sigma_i)_{i=1}^n]\\
%     &= \E[\sup_f|\sum_{i=1}^n \sigma_i^\prime \sum_{\pair{i,j}\in E} w_{i,j}(\sigma_j f(X_i,X_j^\prime)-\\
%     &\qquad\sigma_j f(X_i^{\prime\prime\prime},X_j^{\prime}))|^q\mid \mathcal F^\prime, (\sigma_i)_{i=1}^n]\\
%     &\le 2^q\E[\sup_f|\sum_{i=1}^n \sigma_i^\prime \sum_{\pair{i,j}\in E} w_{i,j}\sigma_j f(X_i,X_j^\prime)|^q\\
%     &\qquad\mid \mathcal F^\prime, (\sigma_i)_{i=1}^n]
% \end{align*}
% and get
% \begin{align*}
%     \E[\sup_f&|\sum_{j=1}^n\sigma_j\sum_{\pair{i,j}\in E}w_{i,j} f(X_i,X_j^\prime)|^q]\\
%     &\le 2^q\E[\sup_f|\sum_{i=1}^n \sum_{\pair{i,j}\in E} w_{i,j}\sigma_i^\prime \sigma_j f(X_i,X_j^\prime)|^q]
% \end{align*}
% Finally,
% \begin{align*}
%     \E[\sup_f&|\sum_{\pair{i,j}\in E} w_{i,j}f(X_i,X_j^\prime)|^q]\\
%     &\le 4^q\E[\sup_f|\sum_{\pair{i,j}\in E} w_{i,j}\sigma_i^\prime \sigma_j f(X_i,X_j^\prime)|^q].    
% \end{align*}
% \end{proof}
% Let $(\sigma_i)_{i=1}^n$ and $(\sigma_i^\prime)_{i=1}^n$ be two independent sequences of i.i.d. Radermacher variables, independent from the $(X_i,X_i^\prime, Y_{i,j}, \tilde{Y}_{i,j})$'s.

% subsection chaining (end)

% \section{LAGRANGIAN DUAL PROBLEM} % (fold)
% \label{sec:lagrangian_dual_problem}

% The Lagrangian function of program (10) is
% \begin{equation}
%     \label{eq:lagrangian_function}
%     \begin{aligned}
%     L(a,b,&\probdistri{},\bmlambda) = a^{1/(1+r)} + b - \sum_{\pair{i,j}\in E}\lambda^{(1)}_{i,j}p_{i,j} \\
%     &+ \sum_{\pair{i,j}\in E}\lambda^{(2)}_{i,j}p_{i,j}\log(1/\delta)-\sum_{\pair{i,j}\in E}\lambda^{(2)}_{i,j}b\\
%     &+ \sum_{i=1}^n \lambda^{(3)}_i\sum_{j:\pair{i,j}\in E}p_{i,j}-\sum_{i=1}^n \lambda^{(3)}_i n_i a\\
%     &+ \sum_{i=1}^n\lambda^{(4)}_i\sqrt{\sum_{j:(i,j)\in E} p^2_{i,j}}(\log(1/\delta))^{1/2}\\
%     &- \sum_{i=1}^n\lambda^{(4)}_i n_i b+ \lambda^{(5)}(\|\probdistri{}\|_2-b)\\
%     &+v(\sum_{\pair{i,j}\in E}p_{i,j}-1)-\lambda^{(6)}a-\lambda^{(7)}b
%     \end{aligned}
% \end{equation}
% where $n_i$ is the number of edges through the vertex $i$ and $L$ is Lagrangian function not risk.
% First, let us minimize the Lagrangian function \eqref{eq:lagrangian_function} which will give the dual function of (10)
% \begin{align*}=
%     g(\bmlambda)&=\inf_{a,b,\probdistri{}} L(a,b,\probdistri{},\lambda)\\
%     & = -v
% \end{align*}
% The minimum of $L(a,b,\probdistri{},\lambda)$ satisfies $\frac{\partial L}{\partial a} = 0$, $\frac{\partial L}{\partial b} = 0$ and $\frac{\partial L}{\partial \probdistri{}} = 0$, that is
% \[\frac{1}{1+r}a^{-r/(1+r)}-\sum_{i=1}^n\lambda_i^{(3)}n_i - \lambda^{(6)} = 0,\]
% \[1 - \sum_{\pair{i,j}\in E}\lambda_{i,j}^{(2)} - \sum_{i=1}^n \lambda_i^{(4)}n_i - \lambda^{(5)} - \lambda^{(7)}=0\]
% and for all $\pair{i,j}\in E$,
% \begin{align*}
%     -\lambda_{i,j}^{(1)} + \lambda_{i,j}^{(2)}\log(1/\delta) + \lambda_i^{(3)} + \lambda_i^{(4)}(\log(1/\delta))^{1/2}p_{i,j}\\(\sum_{k:\pair{i,k}\in E}p_{i,k}^2)^{-1/2}+\lambda^{(5)}p_{i,j}\|\probdistri{}\|_2^{-1}+v=0.
% \end{align*}
% Plugging these equality into \eqref{eq:lagrangian_function}, we will have
% \begin{align*}
%     g(\bmlambda) &= a^{1/(1+r)} - a(\sum_{i=1}^n \lambda_i^{(3)}n_i + \lambda^{(6)})\\
%     &+b - b(\sum_{\pair{i,j}\in E}\lambda_{i,j}^{(2)}+\sum_{i=1}^n\lambda_i^{(4)}n_i +\lambda^{(5)}+\lambda^{(7)})\\
%     &+ \|\probdistri{}\|_2\lambda^{(5)} + \sum_{i=1}^n\lambda_i^{(4)}(\sum_{j:\pair{i,j}\in E}p_{i,j}^2)^{1/2}(\log(1/\delta))^{1/2}\\
%     &+\sum_{\pair{i,j}\in E}p_{i,j}\Big(-\lambda_{i,j}^{(1)} + \lambda_{i,j}^{(2)}\log(1/\delta)+\lambda_i^{(3)}+v\Big)\\
%     &-v\\
%     & = \frac{r}{1+r}a^{1/(1+r)}+\sum_{i=1}^n\lambda_i^{(4)}(\sum_{j:\pair{i,j}\in E}p_{i,j}^2)^{1/2}(\log(1/\delta))^{1/2}\\
%     &+\|\probdistri{}\|_2\lambda^{(5)} - \sum_{\pair{i,j}\in E}p^2_{i,j}\Big(\lambda_i^{(4)}(\log(1/\delta))^{1/2}(\sum_{k:\pair{i,k}\in E}p_{i,k}^2)^{-1/2}\\
%     &+\lambda^{(5)}\|\probdistri{}\|_2^{-1}\Big)+v\\
%     & = \frac{r}{(1+r)^{(1+2r)/r}}(\sum_{i=1}^n \lambda_i^{(3)}n_i + \lambda^{(8)})^{-1/r}\\
%     &+\sum_{i=1}^n\lambda_i^{(4)}(\sum_{j:\pair{i,j}\in E}p_{i,j}^2)^{1/2}(\log(1/\delta))^{1/2}\\
%     &(1-\sum_{j:\pair{i,j}\in E} p^2_{i,j}(\sum_{k:\pair{i,k}\in E}p_{i,k}^2)^{-1})\\
%     &+\|\probdistri{}\|_2\lambda^{(5)}(1-\sum_{\pair{i,j}\in E}p_{i,j}^2\|\probdistri{}\|_2^{-2})-v\\
%     & = \frac{r}{(1+r)^{(1+2r)/r}}(\sum_{i=1}^n \lambda_i^{(3)}n_i + \lambda^{(8)})^{-1/r}-v
% \end{align*}
% \begin{align*}
%     g&(\bmlambda) \\
%     &= \frac{r}{(1+r)^{(1+2r)/r}}(\sum_{i=1}^n \lambda_i^{(3)}n_i + \lambda^{(8)})^{-1/r}-v
% \end{align*}
% under the constrains
% \[\sum_{\pair{i,j}\in E}\lambda_{i,j}^{(2)} + \sum_{i=1}^n \lambda_i^{(4)}n_i + \lambda^{(5)} + \lambda^{(7)}=1,\]
% \[\sum_{i=1}^n\lambda_i^{(3)}+\lambda^{(6)}=0\]
% and for all $\pair{i,j}\in E$,
% \[-\lambda_{i,j}^{(1)}+\lambda_{i,j}^{(2)}\log(1/\delta)+\lambda_i^{(3)}+v=0.\]

% Hence, the Lagrangian dual problem of (10) is
% \begin{align*}
%     \max_{\bmlambda} \quad&-v\\
%     \mbox{s.t.} \quad &\forall \lambda\in \bmlambda{}, \lambda\ge 0\\
%     &\sum_{\pair{i,j}\in E}\lambda_{i,j}^{(2)} + \sum_{i=1}^n \lambda_i^{(4)}n_i + \lambda^{(5)} + \lambda^{(9)}=1\\
%     &\sum_{i=1}^n\lambda_i^{(3)}+\lambda^{(8)}=0\\
%     &\forall\pair{i,j}\in E, -\lambda_{i,j}^{(1)}+\lambda_{i,j}^{(2)}\log(1/\delta)\\
%     &\qquad\qquad\qquad+\lambda_i^{(3)}+\lambda^{(6)}-\lambda^{(7)}=0
% \end{align*}

% section lagrangian_dual_problem (end)

\section{Proofs of the main track} % (fold)
\label{sec:technical_proofs}

\subsection{Proof of Lemma~1} % (fold)
\label{sub:proof_of_lemma_}

Since $R\subset\lebesgue{}_\infty(\xspace{}^2)$ (Assumption~$1$), by Corollary~\ref{cor:covering_number_inequality}, the weighted degenerated $U$-process $\sup_{r\in R}|U_\weight{}(r)|$ can be bounded by the $\lebesgue{}_\infty$ covering number of $R$, that is, for any $\delta\in (0,1/e)$, we have
\[\Pro[\sup_{r\in R}|U_\weight{}(r)|\le \kappa]\ge 1-\delta\]
where 
\begin{align*}
    \kappa\le&\frac{C_1}{\normo{\weight}}\Bigg(\|\weight{}\|_2\int_0^{1}\log N_\infty(\mathcal F, \epsilon)d\epsilon+\max\Big(\|\weight{}\|_2\log(1/\delta)\int_0^{1}\sqrt{\log N_\infty(\mathcal F,\epsilon)}d\epsilon,\\
    &\qquad\qquad(\log(1/\delta))^{3/2}\|\weight{}\|_{\max}, (\log(1/\delta))^2\|\weight{}\|_\infty\Big)\Bigg).
\end{align*}
with a universal constant $C_1<\infty$. 
Then the first inequality for $\sup_{r\in R}|U_\weight{}(r)|$ follows the fact that $R$ satisfies Assumption~$1$.
Similarly, by Lemma~\ref{le:weighted_decoupling_undecoupling} and Lemma~\ref{le:weighted_randomization}, we can convert the moment of $\sup_{r\in R}|\widetilde{U}_\weight{}(r)|$ to the moment of Rademacher chaos
\[4^q\E[\sup_{r\in R}|\sum_{\pair{i,j}\in E}\sigma_i\sigma_j^\prime w_{i,j}\tilde{h}_r(X_i, X_j^\prime,\tilde{Y}_{i,j})|^q]\] 
which can be handled by the by-products of Theorem~\ref{th:moment_inequality_of_weighted_u_statistics}. More specifically, using \eqref{eq:inequality_rademacher_chaos_1} and \eqref{eq:inequality_rademacher_chaos_2} combined with the arguments in Corollary~$1$ and Assumption~$1$ will gives the second inequality for $\sup_{r\in R}|\widetilde{U}_\weight{}(r)|$.

% subsection proof_of_lemma_ (end)

\subsection{Proof of Lemma~2} % (fold)
\label{sub:proofs_of_main_track}
    For any function $r\in\mathcal{R}$, observe first that
    \DoubleColumn
    \begin{align*}
        \E[&q_r(X_1,X_2,Y_{1,2})\mid X_1]\\
        &= \E[\E[q_r(X_1,X_2,Y_{1,2})\mid X_1,X_2]\mid X_1]\\
        &= \E[|1-2\eta(X_1,X_2)|\indicator{}_{r(X_1,X_2)\neq r^*(X_1,X_2)}\mid X_1]
    \end{align*}
    \DoubleColumnEnd
    \SingleColumn
    \begin{align*}
        \E[q_r(X_1,X_2,Y_{1,2})\mid X_1]&= \E[\E[q_r(X_1,X_2,Y_{1,2})\mid X_1,X_2]\mid X_1]\\
        &= \E[|1-2\eta(X_1,X_2)|\indicator{}_{r(X_1,X_2)\neq r^*(X_1,X_2)}\mid X_1]
    \end{align*}
    \SingleColumnEnd
    Then observing that
    \[|1-2\eta(X_1,X_2)|^2\le |1-2\eta(X_1,X_2)|\]
    almost sure, and combining with Jensen inequality, we have
    \begin{align*}
        \Var[\E[q_r(X_1,X_2,Y_{1,2})\mid X_1]] & \le \E[(\E[q_r(X_1,X_2,Y_{1,2})\mid X_1])^2]\\
        &\le \E[|1-2\eta(X_1,X_2)|\indicator{}_{r(X_1,X_2)\neq r^*(X_1,X_2)}]\\
        &= \Lambda(r).
    \end{align*}

\subsection{Proof of Lemma~3} % (fold)
\label{sub:proof_of_lemma_3}
    First, we introduce some notations of weighted ERM of the i.i.d.\ case. 
    We denote $\{\bar{w}_i=\sum_{j:\pair{i,j}\in E}w_{i,j}: i=1,\dots,n\}$ the weights on vertices and introduce the ``loss function''
    \[\gamma(r, X)=2h_r(X)+\Lambda{}(r)\]
    and the weighted empirical loss of vertices
    \[\gamma_\verticeweight{}(r)=\frac{1}{\normo{\verticeweight{}}}\sum_{i=1}^n\bar{w}_i\gamma(r,X_i)=T_\weight{}(r).\]
    Define centered empirical process
    \[\cenprocess{\verticeweight}(r)=\frac{1}{\normo{\verticeweight{}}}\sum_{i=1}^n\bar{w}_i(\gamma(r,X_i)-\Lambda(r))\]
    and the pseudo-distance
    \[d(r,r^\prime) = \frac{\sqrt{\normo{\verticeweight{}}}}{\|\verticeweight\|_2}\left(\E[(\gamma(r,X)-\gamma(r',X))^2]\right)^{1/2}\]
    for every $r,r^\prime\in R$.
    Let $\phi$ be
    \begin{equation}
        \label{eq:proof_phi}
        \phi(\sigma)=12\int_0^{\sigma}(\log N_\infty(R,\epsilon^2))^{1/2}d\epsilon.
    \end{equation}
    From the definition of ``loss function'' $\gamma$, we have the excess risk of $r$ is
    \[\relossf(r,r^*)=\Lambda(r)-\Lambda(r^*)=\Lambda(r).\]
    According to Lemma~$1$, as $\Lambda^2(r)\le \Lambda(r)$, we have for every $r\in R$,
    \begin{align*}
        d(r,r^*)\le \frac{\sqrt{\normo{\verticeweight{}}}}{\|\verticeweight\|_2}\sqrt{5\relossf(r,r^*)}
    \end{align*}
    which implies that the modulus of continuity $\psi$ can be taken as
    \begin{equation}
        \label{eq:proof_psi}
        \psi(\epsilon)=\sqrt{5}\epsilon.
    \end{equation}
    Then from Lemma~\ref{le:maximal_entropy_inequality}, we have
    \[\sqrt{\normo{\verticeweight{}}}\E\left[\sup_{r^\prime\in R,\frac{\|\verticeweight\|_2}{\sqrt{\normo{\verticeweight{}}}}d(r,r^\prime)\le \sigma}|\cenprocess{\verticeweight}(r)-\cenprocess{\verticeweight}(r^\prime)|\right]\le \phi(\sigma).\]
    provided that $\phi(\sigma)/3\le \sqrt{\normo{\verticeweight{}}}\sigma^2$.
    It remains to bound the excess risk of $r_\weight{}$ by the tight bounds for weighted ERM on i.i.d.\ examples by Theorem~\ref{th:weighted_erm_risk_bounds}. For any $\delta\in(0,1)$, we have with probability at least $1-\delta$,
    \DoubleColumn
    \begin{equation}
        \begin{aligned}
            \label{eq:main_result_proof_1}
            \risk(r_\weight{})-\bayeserror \le 2(&\inf_{r\in R}\risk(r)-\bayeserror) + 2\rho \\
            &+ K\log(1/\delta)\epsilon_*^2)
        \end{aligned}
    \end{equation}
    \DoubleColumnEnd
    \SingleColumn
    \begin{equation}
        \begin{aligned}
            \label{eq:main_result_proof_1}
            \risk(r_\weight{})-\bayeserror \le 2(\inf_{r\in R}\risk(r)-\bayeserror) + 2\rho + C\log(1/\delta)\epsilon_*^2)
        \end{aligned}
    \end{equation}
    \SingleColumnEnd
    where $C$ is a universal constant and $\epsilon_*$ is the unique positive solution of the equation
    \[\sqrt{\normo{\verticeweight{}}}\epsilon_*^2=\phi(\psi(\epsilon_*)).\]
    When $ R$ satisfies Assumption~$1$, there exists a universal constant $C^\prime$ such that
    \[\epsilon_*^2\le C^\prime K^{1/(1+\complexbound{})}(\frac{1}{(1-\complexbound{})^2\normo{\verticeweight{}}})^{1/(1+\complexbound{})}\]
    which completes the proof.

% subsection proof_of_lemma_3 (end)

\subsection{Proof of Theorem~1} % (fold)
\label{sub:proof_of_theorem}
    Let us consider the Hoeffding decomposition (5) of $\Lambda_\weight{}(r)$ that is minimized over $r\in R$. The idea of this proof is that the degenerate part $U_\weight{}(r)$ and $\widetilde{U}_\weight{}(r)$ can be bounded by the moment inequality for weighted $U$-processes. Therefore, $r_\weight{}$ is an approximate minimizer of $T_\weight{}(r)$, which can be handled by Lemma~$3$.

    Let $A$ be the event on which
    \[\sup_{r\in R}|U_\weight{}(r)|\le \kappa_A,\]
    where
    \DoubleColumn
    \begin{align*}
    \kappa_A&=\frac{C_A}{1-\complexbound{}}\max\Big(\frac{\|\weight{}\|_2\log(1/\delta)}{\normo{\weight{}}},\\
    &\qquad\frac{\|\weight{}\|_{\max}(\log(1/\delta))^{3/2}}{\normo{\weight{}}}, \frac{\|\weight{}\|_\infty(\log(1/\delta))^2}{\normo{\weight{}}}\Big)
    \end{align*}
    \DoubleColumnEnd
    \SingleColumn
    \begin{align*}
    \kappa_A&=\frac{C_A}{1-\complexbound{}}\max\Big(\frac{\|\weight{}\|_2\log(1/\delta)}{\normo{\weight{}}},\frac{\|\weight{}\|_{\max}(\log(1/\delta))^{3/2}}{\normo{\weight{}}}, \frac{\|\weight{}\|_\infty(\log(1/\delta))^2}{\normo{\weight{}}}\Big)
    \end{align*}
    \SingleColumnEnd
    for an appropriate constant $C_A$. Then by Lemma~$1$, $\Pro[A]\ge 1-\delta/4$. Similarly, let $B$ be the event on which
    \[\sup_{r\in R}|\widetilde{U}_\weight{}(r)|\le \kappa_B.\]
    where
    \DoubleColumn
    \begin{align*}
    \kappa_B&=\frac{C_B}{1-\complexbound{}}\Bigg(\frac{\|\weight{}\|_2}{\normo{\weight{}}}+\max\Big(\frac{\|\weight{}\|_{\max}(\log(1/\delta))^{3/2}}{\normo{\weight{}}},\\
    &\qquad\frac{\|\weight{}\|_\infty(\log(1/\delta))^2}{\normo{\weight{}}}\Big)\Bigg)
    \end{align*}
    \DoubleColumnEnd
    \SingleColumn
    \begin{align*}
    \kappa_B&=\frac{C_B}{1-\complexbound{}}\Bigg(\frac{\|\weight{}\|_2}{\normo{\weight{}}}+\max\Big(\frac{\|\weight{}\|_{\max}(\log(1/\delta))^{3/2}}{\normo{\weight{}}},\frac{\|\weight{}\|_\infty(\log(1/\delta))^2}{\normo{\weight{}}}\Big)\Bigg)
    \end{align*}
    \SingleColumnEnd
    for an appropriate constant $C_B$.
    Then $\Pro[B]\ge 1-\delta/4$.

    By (5), it is clear that, on $A\cap B$, $r_\weight{}$ is a $\rho$-minimizer of $T_\weight{}(r)$ over $r\in R$ in the sense that the value of this latter quantity at its minimum is at most $(\kappa_A+\kappa_B)$ smaller than at $r_\weight{}$. 
    Then, from Lemma~$3$, with probability at least $1-\delta/2$, $r_\weight{}$ is a $(\kappa_A+\kappa_B)$-minimizer of $T_\weight{}(r)$, which the result follows.

\section{Proofs in the appendix} % (fold)
\label{sec:proofs_of_the_appendix}

\subsection{Proof of Theorem \ref{th:fractional_coloring_weighted}} % (fold)
\label{sub:proof_of_theorem_th:fractional_coloring_weighted}

We write, for all $r\in R$,
\begin{equation}
  \label{eq:fractional_coloring_decomposition}
\begin{aligned}
  \risk{}(r)-\empiricalrisk{\weight{}}(r)&\le \sup_{r\in R}\left[\E[\ell(r,z)] - \frac{1}{\normo{\weight{}}}\sum_{i=1}^m w_i\ell(r,z_i)]\right]\\
  &\le \sum_{j=1}^J p_j\left[ \sup_{r\in R}\sum_{i\in\mathcal{C}_j} \frac{w_{k_j^i}}{\normo{\weight{}}} (\E[\ell(r,z)] - \ell(r,z_{k_j^i}))\right].
\end{aligned}
\end{equation}
Now, consider, for each $j$,
\[f_j(S_{\mathcal{C}_j}) = \sup_{r\in R}\sum_{i\in\mathcal{C}_j} \frac{w_{k_j^i}}{\normo{\weight{}}} (\E[\ell(r,z)] - \ell(r,z_{k_j^i})).\]
Let $f$ is defined by, for all training set $S$, $f(S)=\sum_{j=1}^Jp_jf_j(S_{\mathcal{C}_j})$, then $f$ satisfies the conditions of Theorem~\ref{th:fractional_mcdiarmid} with, for any $i\in \set{1,\dots,m}$, $\beta_i\le w_i/\normo{\weight{}}$. Therefore, we can claim that, with probability at least $1-\delta$,
\begin{align*}
  \risk{}(r) - \empiricalrisk{\weight{}}(r)&\le \E\left[\sup_{r\in R}\left[\E[\ell(r,z)] - \frac{1}{\normo{\weight{}}}\sum_{i=1}^m w_i\ell(r,z_i)]\right]\right] + \sqrt{\fcoloring(D_G)\log(1/\delta)}\frac{\|\weight{}\|_2}{\normo{\weight{}}}
\end{align*}
Then, using the standard symmetrization technique \citep[see][Theorem~4]{Usunier2005}, one can bound the first item in the right hand side by $\mathfrak{R}_{\weight{}}^*(R,S)$ which completes the proof. 
% subsection proof_of_theorem_th:fractional_coloring_weighted (end)

\subsection{Proof of Theorem \ref{th:weighted_bennett_inequality}} % (fold)
\label{sub:proof_of_theorem_th:weighted_bennett_inequality}

    We use a auxiliary random variable
    \[\widetilde{Z}=\frac{\normo{\weight{}}}{2b}Z=\sup_{f\in\mathcal{F}}\frac{1}{2b}\sum_{i=1}^nw_i\allowbreak(f(X_i)-\E[f(X_i)]).\]
    We denote by $f_k$ a function such that
    \[f_k = \frac{1}{2b}\sup_{f\in\mathcal F} \sum_{i\neq k}w_i(f(X_i)-\E[f(X_i)]).\]
    We introduce following auxiliary random variables for $k=1,\dots,n$,
    \[Z_k = \frac{1}{2b}\sup_{f\in\mathcal F} \sum_{i\neq k}w_i(f(X_i-\E[f(X_i)]))\]
    and
    \[Z_k^\prime = \frac{1}{2b}w_i(f(X_k)-\E[f(X_k)]).\]
    Denoting by $f_0$ the function achieving the maximum in $Z$, we have
    \[\widetilde{Z}-Z_k\le \frac{1}{2b}w_i(f_0(X_k)-\E[f_0(X_k)])\le 1\ a.s.,\]
    \[\widetilde{Z}-Z_k-Z_k^\prime\ge 0\]
    and
    \[\E[Z_k^\prime] = 0.\]
    The first inequality is derived from $w_i\le 1$ and $\sup_{f\in\mathcal{F}, X\in\mathcal{X}} f(X)-\E[f(X)]\le 2b$.
    Also, we have
    \DoubleColumn
    \begin{align*}
        (n-1)\widetilde{Z} &= \sum_{k=1}^n\frac{1}{2b}\sum_{i\neq k}w_i(f_0(X_i)-\E[f_0(X_i)])\\
        &\le \sum_{k=1}^n Z_k,
    \end{align*}
    \DoubleColumnEnd
    \SingleColumn
    \begin{align*}
        (n-1)\widetilde{Z} &= \sum_{k=1}^n\frac{1}{2b}\sum_{i\neq k}w_i(f_0(X_i)-\E[f_0(X_i)])\\
        &\le \sum_{k=1}^n Z_k,
    \end{align*}
    \SingleColumnEnd
    and
    \begin{align*}
        \sum_{k=1}^n \E_n^k[Z_k^{\prime2}] &= \frac{1}{2b}\sum_{k=1}^n \E[w_i^2(f_k(X_k)-\E[f_k(X_k)])^2]\\
        &\le \frac{1}{4b^2}\|\weight{}\|_2^2\sup_{f\in\mathcal F}\Var[f(X)]\\
        &\le \frac{1}{4b^2}\|\weight{}\|_2^2\sigma^2.
    \end{align*}
    where $\sigma^2\ge \sup_{f\in\mathcal F}\Var[f(X)]$.
    Notice that we use the fact the $X_i$ have identical distribution.
    Applying Theorem 1 of \cite{Bousquet2002a} with $v=2\E[\widetilde{Z}]+\frac{\|\weight{}\|_2^2}{4b^2}\sigma^2$ will give
    \[\Pro[\widetilde{Z}-\E[\widetilde{Z}]\ge \sqrt{2vx}+\frac{x}{3}]\le e^{-x},\]
    and then
    \DoubleColumn
    \begin{align*}
        \Pro[\frac{\normo{\weight{}}}{2b}(Z-\E[Z])&\ge \sqrt{2x(\frac{\normo{\weight{}}}{b}\E[Z]+\frac{\|\weight{}\|_2^2}{4b^2}\sigma^2)}\\
        &\qquad+\frac{x}{3}] \le e^{-x}
    \end{align*}
    \DoubleColumnEnd
    \SingleColumn
    \begin{align*}
        \Pro\left[\frac{\normo{\weight{}}}{2b}(Z-\E[Z])\ge \sqrt{2x(\frac{\normo{\weight{}}}{b}\E[Z]+\frac{\|\weight{}\|_2^2}{4b^2}\sigma^2)}+\frac{x}{3}\right] \le e^{-x}
    \end{align*}
    \SingleColumnEnd
    which proves the inequality.
% The following inequality is Lemma A.5 of \citep{Massart2006}.
% \begin{lemma}[\citep{Massart2006}]
%     \label{le:weighted_process_inequality}
%     Let $S$ be a countable set, $u\in S$ and $a:S\to \mathbb R_+$ such that $a(u)=\inf_{t\in S} a(t)$. Let $Z$ be a process indexed by $S$ and assume that the nonnegative random variable $\sup_{t\in\mathcal B(\epsilon)}[Z(u)-Z(t)]$ has finite expectation for any positive number $\epsilon$, where $\mathcal B(\epsilon)=\{t\in S,a(t)\le \epsilon\}$. Let $\varphi$ be a nonnegative function on $\mathbb R_+$ such that $\varphi(x)/x$ is nonincreasing on $\mathbb R_+$ and satisfies for some positive number $\epsilon_*$,
%     \[\E[\sup_{t\in \mathcal B(\epsilon)}[Z(u)-Z(t)]]\le \varphi(\epsilon), \forall \epsilon\ge \epsilon_*.\]
%     Then, one has, for any positive number $x\ge \epsilon_*$,
%     \[\E[\sup_{t\in S}\frac{Z(u)-Z(t)}{a^2(t)+x^2}]\le 4x^{-2}\varphi(x).\]
% \end{lemma}

\subsection{Proof of Theorem \ref{th:weighted_erm_risk_bounds}} % (fold)
\label{sub:proof_of_theorem_th:weighted_erm_risk_bounds}

    Since $R$ satisfies Condition \ref{con:separability_condition}, we notice that, by dominated convergence, for every $r\in R$, considering the sequence $\{r_k\}$ provided by Condition \ref{con:separability_condition}, one has $\Pro[\gamma(\cdot,r_k)]$ that tends to $\Pro[\gamma(\cdot, r)]$ as $k$ tends to infinity. Denote the bias term of loss $\relossf(r^*,R)=\inf_{r\in R}\relossf(r^*,r)$. Hence, $\relossf(r^*,R)=\relossf(r^*,R^\prime)$, which implies that there exists some point $\pi(r^*)$ (which may depend on $\epsilon_*$) such that $\pi(r^*)\in R^\prime$ and
    \begin{equation}
        \label{eq:asymptotic_function_inequality}
        \relossf(r^*, \pi(r^*))\le \relossf(r^*,R)+\epsilon_*^2.
    \end{equation}

    We start from the identity
    \SingleColumn
    \[\relossf(r^*,\hat{r})=\relossf(r^*,\pi(r^*))+\gamma_\weight(\hat{r})-\gamma_\weight(\pi(r^*))+\cenprocess{\weight}(\pi(r^*))-\cenprocess{\weight}(\hat{r})\]
    \SingleColumnEnd
    \DoubleColumn
    \begin{align*}
        \relossf(r^*,\hat{r})&=\relossf(r^*,\pi(r^*))+\gamma_\weight(\hat{r})-\gamma_\weight(\pi(r^*))\\
        &\qquad+\cenprocess{\weight}(\pi(r^*))-\cenprocess{\weight}(\hat{r})
    \end{align*}
    \DoubleColumnEnd
    which, by definition of $\hat{r}$, implies that
    \[\relossf(r^*,\hat{r})\le \rho+\relossf(r,\pi(r^*))+\cenprocess{\weight}(\pi(r^*))-\cenprocess{\weight}(\hat{r}).\]
    Let $x=\sqrt{K^\prime y}\epsilon_*$, where $K^\prime$ is a constant to be chosen later such that $K^\prime\ge 1$ and
    \[V_x=\sup_{r\in R}\frac{\cenprocess{\weight}(\pi(r^*))-\cenprocess{\weight}(r)}{\relossf(r^*,r)+\epsilon^*+x^2}.\]
    Then,
    \[\relossf(r^*,\hat{r})\le \rho+\relossf(r^*,\pi(r^*))+V_x(\relossf(r^*,\hat{r})+x^2+\epsilon_*^2)\]
    and therefore, on the event $V_x<1/2$, one has
    \[\relossf(r^*,\hat{r})\le 2(\rho+\relossf(r^*,\pi(r^*)))+\epsilon_*^2+x^2,\]
    yielding
    \SingleColumn
    \begin{equation}
        \label{eq:risk_bounds_convert}
        \Pro[\relossf(r^*,\hat{r})\le 2(\rho+\relossf(r^*,\pi(r^*)))+3\epsilon_*^2+x^2]\le \Pro[V_x\ge \frac{1}{2}].
    \end{equation}
    \SingleColumnEnd
    \DoubleColumn
    \begin{equation}
        \begin{aligned}
            \label{eq:risk_bounds_convert}
            \Pro[\relossf(r^*,\hat{r})\le 2(\rho+\relossf(r^*,\pi(r^*)))+3\epsilon_*^2+x^2]\\
            \le \Pro[V_x\ge \frac{1}{2}].
        \end{aligned}
    \end{equation}
    \DoubleColumnEnd
    Since $\relossf$ is bounded by 1, we may always assume $x$ (and thus $\epsilon_*$) to be not larger than 1. Assuming that $x\le 1$, it remains to control the variable $V_x$ via Theorem \ref{th:weighted_bennett_inequality}. In order to use Theorem \ref{th:weighted_bennett_inequality}, we first remark that, by Condition \ref{con:separability_condition},
    \[V_x=\sup_{r\in R^\prime}\frac{\cenprocess{\weight}(\pi(r^*))-\cenprocess{\weight}(r)}{\relossf(r^*,r)+\epsilon^*+x^2}\]
    which means that we indeed have to deal with a countably indexed empirical process. Note that the triangle inequality implies via \eqref{eq:variancD_Eistance_inequality}, \eqref{eq:asymptotic_function_inequality} and \eqref{eq:weighted_erm_risk_bound_con1} that
    \begin{equation}
        \begin{aligned}
            \label{eq:bound_of_variance}
            (\Var[\gamma(r,\cdot)-\gamma(\pi(r^*),\cdot)])^{\frac{1}{2}}&\le d(r^*,r)+d(r^*, \pi(r^*))\\
            &\le 2\frac{\sqrt{\normo{\weight{}}}}{\|\weight{}\|_2}\psi(\sqrt{\relossf(r^*,r)+\epsilon_*^2})
        \end{aligned}
    \end{equation}
    Since $\gamma$ takes its values in $[0,1]$, introducing the functions $\psi_1=\min(1,2\psi)$ and, we derive from \eqref{eq:bound_of_variance} that
    \DoubleColumn
    \begin{align*}
        \sup_{r\in R}\Var[\frac{\cenprocess{\weight}(\pi(r^*))-\cenprocess{\weight}(t)}{\relossf(r^*,t)+\epsilon^*+x^2}]&\le \sup_{\epsilon\ge 0}\frac{(\frac{\sqrt{\normo{\weight{}}}}{\|\weight{}\|_2}\psi_1(\epsilon))^2}{(\epsilon^2+x^2)^2}\\
        &\le \frac{\normo{\weight{}}}{\|\weight{}\|_2^2x^2}\sup_{\epsilon\ge 0}(\frac{\psi_1(\epsilon)}{\max(\epsilon,x)})^2.
    \end{align*}
    \DoubleColumnEnd
    \SingleColumn
    \[\sup_{r\in R}\Var[\frac{\cenprocess{\weight}(\pi(r^*))-\cenprocess{\weight}(r)}{\relossf(r^*,r)+\epsilon^*+x^2}]\le \frac{\normo{\weight{}}}{\|\weight{}\|_2^2}\sup_{\epsilon\ge 0}\frac{\psi_1^2(\epsilon)}{(\epsilon^2+x^2)^2}\le \frac{\normo{\weight{}}}{\|\weight{}\|_2^2x^2}\sup_{\epsilon\ge 0}(\frac{\psi_1(\epsilon)}{\max(\epsilon,x)})^2.\]
    \SingleColumnEnd
    Now the monotonicity assumptions on $\psi$ imply that either $\psi(\epsilon)\le \psi(x)$ if $x\ge \epsilon$ or $\psi(\epsilon)/\epsilon\le \psi(x)/x$ if $x\le \epsilon$. Hence, one has in any case $\psi(\epsilon)/(\max(\epsilon,x))\le \psi(x)/x$, which finally yields
    \[\sup_{r\in R}\Var[\frac{\cenprocess{\weight}(\pi(r^*))-\cenprocess{\weight}(r)}{\relossf(r^*,r)+\epsilon^*+x^2}]\le\frac{\normo{\weight{}}\psi_1^2(x)}{\|\weight{}\|_2^2x^4}.\]
    On the other hand, since $\gamma$ takes its values in $[0,1]$, we have
    \[\sup_{r\in R}\|\frac{\gamma(r,\cdot)-\gamma(\pi(r^*),\cdot)}{\relossf(r^*,r)+x^2}\|_\infty\le \frac{1}{x^2}.\]
    We can therefore apply Theorem~\ref{th:weighted_bennett_inequality} with $v=\psi_1^2(x)x^{-4}$ and $b=x^{-2}$, which gives that, on a set $\Omega_y$ with probability larger than $1-\exp(-y)$, the inequality
    \begin{equation}
        \begin{aligned}
            \label{eq:v_x_bennett_inequality}
            V_x < \E[V_x]+\sqrt{\frac{2y(\psi_1^2(x)x^{-2}+4\E[V_x])}{\normo{\weight{}}x^2}}+\frac{2y}{3\normo{\weight{}}x^2}.
        \end{aligned}
    \end{equation}
    Now since $\epsilon_*$ is assumed to be not larger than 1, one has $\psi(\epsilon_*)\ge \epsilon_*$ and therefore, for every $\sigma\ge \psi(\epsilon_*)$, the following inequality derives from the definition of $\epsilon_*$ by monotonicity:
    \[\frac{\phi(\sigma)}{\sigma^2}\le \frac{\phi(\psi(\epsilon_*))}{w^2(\epsilon_*)}\le \frac{\phi(\psi(\epsilon_*))}{\epsilon_*^2}=\sqrt{\normo{\weight{}}}.\]
    Thus, \eqref{eq:weighted_erm_risk_bound_con2} holds for every $\sigma\ge \psi(\epsilon_*)$. In order to control $\E[V_x]$, we intend to use Lemma A.5 of \cite{Massart2006}. 
    For every $r\in R^\prime$, we introduce $a^2(r)=\max(\relossf(r^*,\pi(r^*)),\relossf(r^*,r))$. Then by \eqref{eq:asymptotic_function_inequality}, $\relossf(r^*,r)\le a^2(r)\le \relossf(r^*,r)+\epsilon_*^2$. Hence, we have, on the one hand, that
    \[\E[V_x]\le \E[\sup_{r\in R^\prime}\frac{\cenprocess{\weight}(\pi(r^*))-\cenprocess{\weight}(r)}{a^2(r)+x^2}].\]
    and, on the other hand, that, for every $\epsilon\ge \epsilon_*$,
    \SingleColumn
    \[\E[\sup_{r\in R^\prime,a(r)\le \epsilon}\cenprocess{\weight}(\pi(r^*))-\cenprocess{\weight}(r)]\le \E[\sup_{r\in R^\prime,\relossf(r^*,r)\le \epsilon^2}\cenprocess{\weight}(\pi(r^*))-\cenprocess{\weight}(r)].\]
    \SingleColumnEnd
    \DoubleColumn
    \begin{align*}
        \E[\sup_{r\in R^\prime,a(r)\le \epsilon}&\cenprocess{\weight}(\pi(r^*))-\cenprocess{\weight}(r)]\\
        &\le \E[\sup_{r\in R^\prime,\relossf(r^*,r)\le \epsilon^2}\cenprocess{\weight}(\pi(r^*))-\cenprocess{\weight}(r)].
    \end{align*}
    \DoubleColumnEnd
    Now by \eqref{eq:asymptotic_function_inequality} if there exists some $r\in R^\prime$ such that $\relossf(r^*,r)\le \epsilon^2$, then $\relossf(r^*,\pi(r^*))\le \epsilon^2+\epsilon_*^2\le 2\epsilon^2$ and therefore, by assumption \eqref{eq:weighted_erm_risk_bound_con1} and monotonicity of $\theta\to \psi(\theta)/\theta$, $d(\pi(r^*),r)\le 2\frac{\sqrt{\normo{\weight{}}}}{\|\weight{}\|_2}\psi(\sqrt{2}\epsilon)\le 2\sqrt{2}\frac{\sqrt{\normo{\weight{}}}}{\|\weight{}\|_2}\psi(\epsilon)$, then $\frac{\|\weight{}\|_2}{\sqrt{\normo{\weight{}}}}d(\pi(r^*),r)\le 2\sqrt{2}\psi(\epsilon)$. Thus, we derive from \eqref{eq:weighted_erm_risk_bound_con2} that, for every $\epsilon\ge \epsilon_*$,
    \[\E[\sup_{r\in R^\prime,\relossf(r^*,r)\le \epsilon^2}\cenprocess{\weight}(\pi(r^*))-\cenprocess{\weight}(r)]\le \phi(2\sqrt{2}\psi(\epsilon))\]
    and since $\theta\to \phi(2\sqrt{2}\psi(\theta))/\theta$ is nonincreasing, we can use Lemma A.5 of \cite{Massart2006} to get
    \[\E[V_x]\le 4\phi(2\sqrt{2}\psi(x))/(\sqrt{\normo{\weight{}}}x^2),\]
    and by monotonicity of $\theta\to\phi(\theta)/\theta$,
    \[\E[V_x]\le 8\sqrt{2}\phi(\psi(x))/(\sqrt{\normo{\weight{}}}x^2).\]
    Thus, using the monotonicity of $\theta\to \phi(\psi(\theta))/\theta$, and the definition of $\epsilon_*$, we derive that
    \begin{equation}
        \label{eq:bound_of_expectation}
        \E[V_x]\le \frac{8\sqrt{2}\phi(\psi(\epsilon_*))}{\sqrt{\normo{\weight{}}}x\epsilon_*}=\frac{8\sqrt{2}\epsilon_*}{x}\le \frac{8\sqrt{2}}{\sqrt{K^\prime y}}\le \frac{8\sqrt{2}}{\sqrt{K^\prime}},
    \end{equation}
    provided that $x\ge \epsilon_*$, which holds since $K^\prime\ge 1$. Now, the monotonicity of $\theta\to\psi_1(\theta)/\theta$ implies that $x^{-2}\psi_1^2(x)\le \epsilon_*^{-2}\psi_1^2(\epsilon_*)$, but since $\phi(\theta)/\theta\ge \phi(1)\ge 1$ for every $\theta\in [0,1]$, we derive from \eqref{eq:risk_bound_con3} and the monotonicity of $\phi$ and $\theta\to\phi(\theta)/\theta$ that
    \[\frac{\psi_1^2(\epsilon_*)}{\epsilon_*^2}\le \frac{\phi^2(\psi_1(\epsilon_*))}{\epsilon_*^2}\le \frac{\phi^2(2\psi(\epsilon_*))}{\epsilon_*^2}\le 4\frac{\phi^2(\psi(\epsilon_*))}{\epsilon_*^2}\]
    and, therefore, $x^{-2}\psi_1^2(x)\le 4\normo{\weight{}}\epsilon_*^2$. Plugging this inequality together with \eqref{eq:bound_of_expectation} into \eqref{eq:v_x_bennett_inequality} implies that, on the set $\Omega_y$,
    \[V_x < \frac{8\sqrt{2}}{\sqrt{K^\prime}}+\sqrt{\frac{2y(4\normo{\weight{}}\epsilon_*^2+32/\sqrt{K^\prime})}{\normo{\weight{}}x^2}}+\frac{2y}{3\normo{\weight{}}x^2}.\]
    It remains to replace $x^2$ by its value $K^\prime y\epsilon_*^2$ to derive that, on the set $\Omega_y$, the following inequality holds:
    \[V_x < \frac{8\sqrt{2}}{\sqrt{K^\prime}}+\sqrt{\frac{8(1+4(\normo{\weight{}}\epsilon_*^2\sqrt{K^\prime})^{-1})}{K^\prime}}+\frac{2}{3\normo{\weight{}}K^\prime\epsilon_*^2}.\]
    Taking into account that $\phi(\psi(\theta))\ge \phi(\min(1,\psi(\theta)))\ge \theta$ for every $\theta\in [0,1]$, we deduce from the definition of $\epsilon_*$ that $\normo{\weight{}}\epsilon_*^2\ge 1$ and, therefore, the preceding inequality becomes, on $\Omega_y$,
    \[V_x < \frac{8\sqrt{2}}{\sqrt{K^\prime}}+\sqrt{\frac{8(1+4/\sqrt{K^\prime})}{K^\prime}}+\frac{2}{3K^\prime}.\]
    Hence, choosing $K^\prime$ as a large enough numerical constant guarantee that $V_x < 1/2$ on $\Omega_y$ and, therefore, \eqref{eq:risk_bounds_convert} yields
    \SingleColumn
    \begin{equation}
        \label{eq:risk_bounds_convert}
        \Pro[\relossf(r^*,\hat{r})\le 2(\rho+\relossf(r^*,\pi(r^*)))+3\epsilon_*^2+x^2]\le \Pro[\Omega_y^c]\le e^{-y}.
    \end{equation}
    \SingleColumnEnd
    \DoubleColumn
        \begin{align*}
            \label{eq:risk_bounds_convert}
            \Pro[\relossf(r^*,\hat{r})\le 2(\rho+\relossf(r^*,\pi(r^*)))&+3\epsilon_*^2+x^2]\\
            &\le \Pro[\Omega_y^c]\\
            &\le e^{-y}.
        \end{align*}
    \DoubleColumnEnd
    We get the required probability bound \eqref{th:weighted_erm_risk_bounds} by setting $K=K^\prime+3$.

\subsection{Proof of Lemma \ref{le:maximal_entropy_inequality}} % (fold)
\label{sub:proof_of_lemma_le:maximal_entropy_inequality}

    We first perform the control of $\E[\sup_{f\in\mathcal{F}}v_\weight(f-f_0)]$. For simplicity, we denote $H_\infty(\mathcal{F}, \epsilon)=\log N_\infty(\mathcal{F},\epsilon)$. For any integer $j$, we set $\sigma_j=\sigma2^{-j}$ and $H_j=H_\infty(\mathcal{F},\sigma_j^2)$. By definition of $H_j=H_\infty(\mathcal{F},\sigma_j^2)$, for any integer $j\ge 1$, we can define a mapping $\Pi_j$ from $\mathcal{F}$ to some finite collection of functions such that
    \begin{equation}
        \label{eq:bounded_entropy_mapping}
        \log \#\{\Pi_j\mathcal{F}\}\le H_j
    \end{equation}
    and
    \begin{equation}
        \label{eq:bounded_entropy_condition}
        \Pi_jf\le f \text{ with }P(f-\Pi_jf)\le \sigma_j^2, \forall f\in\mathcal{F}.
    \end{equation}
    For $j=0$, we choose $\Pi_0$ to be identically equal to $f_0$. For this choice of $\Pi_0$, we still have
    \begin{equation}
        \label{eq:bounded_entropy_f0}
        P(|f-\Pi_0f|)=P[|f-f_0|]\le \sigma_0^2=\sigma
    \end{equation}
    for every $f\in\mathcal{F}$. Furthermore, since we may always assume that the extrmities of the balls used to cover $\mathcal{F}$ take their values in $[0,1]$, we also have for every integer $j$ that
    \[0\le\Pi_jf\le1.\]
    Noticing that since $u\to H_\infty(\mathcal{F},u^2)$ is nonincreasing,
    \[H_1\le \sigma_1^{-2}\varphi^2(\sigma),\]
    and under the condition $4\varphi(\sigma)\le \sigma^2\sqrt{\normo{\weight{}}}$, one has $H_1\le \sigma_1^2\normo{\weight{}}$. Thus, since $j\to H_j\sigma_j^{-2}$ increases to infinity, the set $\{j\ge 0: H_j\le \sigma_j^2\normo{\weight{}}\}$ is a nonvoid interval of the form
    \[\{j\ge 0: H_j\le \sigma_j^2\normo{\weight{}}\}=[0,J],\]
    with $J\ge 1$. For every $f\in\mathcal{F}$, starting from the decomposition
    \DoubleColumn
    \begin{align*}
    -v_\weight(f)=\sum_{j=0}^{J-1}v_\weight(\Pi_jf)-v_\weight(\Pi_{j+1}f)\\
    +v_\weight(\Pi_Jf)-v_\weight(f),
    \end{align*}
    \DoubleColumnEnd
    \SingleColumn
    \begin{align*}
    -v_\weight(f)=\sum_{j=0}^{J-1}v_\weight(\Pi_jf)-v_\weight(\Pi_{j+1}f)+v_\weight(\Pi_Jf)-v_\weight(f),
    \end{align*}
    \SingleColumnEnd
    we derive, since $\Pi_J(f)\le f$ and $P(f-\Pi_j(f))\le \sigma_J^2$, that
    \[-v_\weight(f)=\sum_{j=0}^{J-1}v_\weight(\Pi_jf)-v_\weight(\Pi_{j+1}f)+\sigma_J^2\]
    and, therefore,
    \DoubleColumn
    \begin{equation}
        \begin{aligned}
            \label{eq:bounded_entropy_decomposition}
            \E[&\sup_{f\in\mathcal{F}}[-v_\weight(f)]]\\
            &\le \sum_{j=0}^{J-1}\E[\sup_{f\in\mathcal{F}}v_\weight(\Pi_jf)-v_\weight(\Pi_{j+1}f)]+\sigma_J^2.
        \end{aligned}
    \end{equation}
    \DoubleColumnEnd
    \SingleColumn
    \begin{equation}
        \begin{aligned}
            \label{eq:bounded_entropy_decomposition}
            \E[\sup_{f\in\mathcal{F}}[-v_\weight(f)]]\le \sum_{j=0}^{J-1}\E[\sup_{f\in\mathcal{F}}v_\weight(\Pi_jf)-v_\weight(\Pi_{j+1}f)]+\sigma_J^2.
        \end{aligned}
    \end{equation}
    \SingleColumnEnd
    Now, it follows from \eqref{eq:bounded_entropy_condition} and \eqref{eq:bounded_entropy_f0} that, for every integer $j$ and every $f\in\mathcal{F}$, one has
    \[P[|\Pi_jf-\Pi_{j+1}f|]\le \sigma_j^2+\sigma_{j+1}^2=5\sigma_{j+1}^2\]
    and, therefore, since $|\Pi_jf-\Pi_{j+1}f|\le 1$,
    \[P[|\Pi_jf-\Pi_{j+1}f|^2]\le 5\sigma_{j+1}^2.\]
    Moreover, \eqref{eq:bounded_entropy_mapping} ensures that the number of functions of the form $\Pi_jf-\Pi_{j+1}f$ when varies in $\mathcal{F}$ is not larger than $\exp(H_j+H_{j+1})\le \exp(2H_{j+1})$, Hence, we derive from the maximal inequality for random vectors \citep[see][Lemma A.1]{Massart2006} and the by-product of the proof of Bernstein's inequality for the weighted sum of networked random variables \citep[see][Lemma 16]{ramon2015learning} that
    \DoubleColumn
    \begin{align*}
        \sqrt{\normo{\weight{}}}&\E[\sup_{f\in\mathcal{F}}[v_\weight(\Pi_jf)-v_\weight(\Pi_{j+1}f)]]\\
        &\le 2[\sigma_{j+1}\sqrt{5H_{j+1}}+\frac{1}{3\sqrt{\normo{\weight{}}}}H_{j+1}]\\
    \end{align*}
    \DoubleColumnEnd
    \SingleColumn
    \begin{align*}
        \sqrt{\normo{\weight{}}}\E[\sup_{f\in\mathcal{F}}[v_\weight(\Pi_jf)-v_\weight(\Pi_{j+1}f)]]\le 2[\sigma_{j+1}\sqrt{5H_{j+1}}+\frac{1}{3\sqrt{\normo{\weight{}}}}H_{j+1}]\\
    \end{align*}
    \SingleColumnEnd
    because $w_i\le 1, \forall i\in 1,\dots,n$,
    and \eqref{eq:bounded_entropy_decomposition} becomes
    \DoubleColumn
    \begin{equation}
    \begin{aligned}
        \label{eq:bounded_entropy_result}
        &\sqrt{\normo{\weight{}}}\E[\sup_{f\in\mathcal{F}}-v_\weight(f)]\\
        &\le 2\sum_{j=1}^J[\sigma_j\sqrt{5H_j}+\frac{1}{3\sqrt{\normo{\weight{}}}}H_j]+4\sqrt{\normo{\weight{}}}\sigma_{J+1}^2.
    \end{aligned}
    \end{equation}
    \DoubleColumnEnd
    \SingleColumn
    \begin{equation}
    \begin{aligned}
        \label{eq:bounded_entropy_result}
        \sqrt{\normo{\weight{}}}\E[\sup_{f\in\mathcal{F}}-v_\weight(f)]\le 2\sum_{j=1}^J[\sigma_j\sqrt{5H_j}+\frac{1}{3\sqrt{\normo{\weight{}}}}H_j]+4\sqrt{\normo{\weight{}}}\sigma_{J+1}^2.
    \end{aligned}
    \end{equation}
    \SingleColumnEnd
    It follows from the definition of $J$ that, on the one hand, for every $j\le J$,
    \[\frac{1}{3\sqrt{\normo{\weight{}}}}H_j\le \frac{1}{3}\sqrt{H_j}\]
    and, on the other hand,
    \[4\sqrt{\normo{\weight{}}}\sigma_{J+1}^2\le 4\sigma_{J+1}\sqrt{H_{j+1}}.\]
    Hence, plugging these inequalities in \eqref{eq:bounded_entropy_result} yields
    \[\sqrt{\normo{\weight{}}}\E[\sup_{f\in\mathcal{F}}-v_\weight(f)]\le 6\sum_{j=1}^{J+1}\sigma_j\sqrt{H_j},\]
    and the result follows. The control of $\E[\sup_{f\in\mathcal{F}}v_\weight(f-f_0)]$ can be performed analogously.
    % , changing lower into upper approximations in the dyadic approximation scheme described above.

\subsection{Proof of Lemma \ref{le:weighted_decoupling_undecoupling}} % (fold)
\label{sub:proof_of_lemma_le:weighted_decoupling_undecoupling}

    This Lemma is derived from Lemma \ref{le:weighted_u_statistics_decoupling}, thus we can follow the similar arguments that can be found in \cite{de2012decoupling}.

    For any random variable $X$, we denote by $\mathcal L(X)$ its distribution. We denote by $\Sigma$ (respectively $\Sigma^\prime$) the sigma-filed generated by $\{X_1,\dots,X_n\}$ (respectively $\{X_1^\prime,\dots,X_n^\prime\}$). Let $(Y_{i,j}^\prime)_{(i,j)\in E}$ be Bernoulli random variables such that $\Pro[Y_{i,j}'=1\mid \Sigma, \Sigma^\prime]=\eta(X_i^\prime, X_j^\prime)$. Let $(\sigma_i)_{i=1}^n$ be independent Rademacher variables and define:
    \begin{align*}
        Z_i = X_i \text{ if } \sigma_i = 1 \text{, and } X_i^\prime \text{ otherwise,}\\
        Z_i^\prime = X_i^\prime \text{ if } \sigma_i = 1 \text{, and } X_i^\prime \text{ otherwise.}
    \end{align*}
    Conditionally upon the $X_i$ and $X_i^\prime$, the random vector $(Z_i, Z_i^\prime)$ takes the values $(X_i, X_i^\prime)$ or $(X_i^\prime, X_i)$, each with probability 1/2. In particular, we have:
    \SingleColumn
    \begin{equation}
        \label{eq:weighted_u_statistics_decoupling_fact1}
        \mathcal L(X_1,\dots,X_n,X_1^\prime,\dots,X_n^\prime) = \mathcal L(Z_1,\dots,Z_n,Z_1^\prime,\dots,Z_n^\prime).
    \end{equation}
    \SingleColumnEnd
    \DoubleColumn
    \begin{equation}
        \begin{aligned}
            \label{eq:weighted_u_statistics_decoupling_fact1}
            \mathcal L(X_1,\dots,X_n,X_1^\prime,\dots,X_n^\prime) =\\ \mathcal L(Z_1,\dots,Z_n,Z_1^\prime,\dots,Z_n^\prime)
        \end{aligned}
    \end{equation}
    \DoubleColumnEnd
    and
    \begin{equation}
        \label{eq:weighted_u_statistics_decoupling_fact2}
        \mathcal L(X_1,\dots,X_n) = \mathcal L(Z_1,\dots,Z_n).
    \end{equation}
    Let $\set{\tilde{Y}_{i,j}}_{\pair{i,j}\in E}$ be Bernoulli random variables such that $\Pro[\tilde{Y}_{i,j}=1 \mid \Sigma,\Sigma'] = \eta(X_i, X_j')$ and define for $\pair{i,j}\in E$,
    \begin{equation*}
        \hat{Y}_{i,j}=\begin{cases}
            Y_{i,j}&\text{if } \sigma_i=1\text{ and } \sigma_j=-1\\
            Y_{i,j}^\prime&\text{if } \sigma_i=-1\text{ and } \sigma_j=1\\
            \tilde{Y}_{i,j}&\text{if } \sigma_i=1\text{ and } \sigma_j=1\\
            \tilde{Y}_{i,j}&\text{if } \sigma_i=-1\text{ and } \sigma_j=-1\\
        \end{cases}
    \end{equation*}
    Notice that for all $f$,
    \DoubleColumn
    \begin{align*}
        \E_\sigma[&\tilde{h}_r(Z_i,Z_j^\prime,\tilde{Y}_{i,j})]\\
        &=\frac{1}{4}(\tilde{h}_r(X_i,X_j, Y_{i,j})+\tilde{h}_r(X_i^\prime,X_j,\tilde{Y}_{i,j})\\
        &\qquad+\tilde{h}_r(X_i,X_j^\prime,\tilde{Y}_{i,j})+\tilde{h}_r(X_i^\prime,X_j^\prime,Y_{i,j}^\prime))    
    \end{align*}
    \DoubleColumnEnd
    \SingleColumn
    \begin{align*}
        \E_\sigma[\tilde{h}_r(Z_i,Z_j^\prime,\hat{Y}_{i,j})]=\frac{1}{4}(\tilde{h}_r(X_i,X_j, Y_{i,j})+\tilde{h}_r(X_i^\prime,X_j,\tilde{Y}_{i,j})+\tilde{h}_r(X_i,X_j^\prime,\tilde{Y}_{i,j})+\tilde{h}_r(X_i^\prime,X_j^\prime,Y_{i,j}^\prime))    
    \end{align*}
    \SingleColumnEnd
    where $\E_\sigma$ denotes the expectation taken with respect to $\{\sigma_i\}_{i=1}^n$. Moreover, using
    \[\E[\tilde{h}_r(X_i^\prime,X_j^\prime,Y_{i,j}^\prime)\mid \Sigma]=0\]
    and (degenerated)
    \begin{align*}
        \E[\tilde{h}_r(X_i,X_j^\prime,\tilde{Y}_{i,j})\mid \Sigma] = 0\\
         \E[\tilde{h}_r(X_i^\prime,X_j,\tilde{Y}_{i,j})\mid \Sigma] = 0
    \end{align*}
    we easily get
    \[\tilde{h}_r(X_i,X_j,Y_{i,j})=4\E[\tilde{h}_r(Z_i,Z_j^\prime,\hat{Y}_{i,j})\mid \Sigma]\]
    For all $q\ge 1$, we therefore have
    \DoubleColumn
    \begin{align*}
        \E[&\sup_{r\in R}|\sum_{\pair{i,j}\in E}w_{i,j}\tilde{h}_r(X_i,X_j,Y_{i,j})|^q]\\
        &=\E[\sup_{r\in R}|\sum_{\pair{i,j}\in E}4w_{i,j}\E[\tilde{h}_r(Z_i,Z_j^\prime,\tilde{Y}_{i,j})\mid\Sigma]|^q]\\
        &\le 4^q\E[\sup_{r\in R}|\sum_{\pair{i,j}\in E}w_{i,j}\tilde{h}_r(Z_i,Z_j^\prime,\tilde{Y}_{i,j})|^q]
    \end{align*}
    \DoubleColumnEnd
    \SingleColumn
    \begin{align*}
        \E[\sup_{r\in R}|\sum_{\pair{i,j}\in E}w_{i,j}\tilde{h}_r(X_i,X_j,Y_{i,j})|^q]&=\E[\sup_{r\in R}|\sum_{\pair{i,j}\in E}4w_{i,j}\E[\tilde{h}_r(Z_i,Z_j^\prime,\hat{Y}_{i,j})\mid\Sigma]|^q]\\
        &\le 4^q\E[\sup_{r\in R}|\sum_{\pair{i,j}\in E}w_{i,j}\tilde{h}_r(Z_i,Z_j^\prime,\hat{Y}_{i,j})|^q]
    \end{align*}
    \SingleColumnEnd
    derived from the facts that the supreme and $|x|^p (p\ge1)$ are convex functions and the Jansen inequality. According to \eqref{eq:weighted_u_statistics_decoupling_fact1} and the fact that the distribution of $\hat{Y}_{i,j}$ only depends on the realization $Z_i,Z_j^\prime$, i.e.\ $\Pro[\hat{Y}_{i,j}\mid Z_i,Z_j^\prime]=\eta(Z_i,Z_j^\prime)$, we obtain
    \DoubleColumn
    \begin{align*}
        4^q\E[&\sup_{r\in R}|\sum_{\pair{i,j}\in E}w_{i,j}\hat{h}_r(Z_i,Z_j^\prime,\tilde{Y}_{i,j})|^q]\\
        &= 4^q\E[\sup_{r\in R}|\sum_{\pair{i,j}\in E}w_{i,j}\tilde{h}_r(X_i,X_j^\prime,\tilde{Y}_{i,j})|^q]
    \end{align*}
    \DoubleColumnEnd
    \SingleColumn
    \begin{align*}
        4^q\E[\sup_{r\in R}|\sum_{\pair{i,j}\in E}w_{i,j}\tilde{h}_r(Z_i,Z_j^\prime,\hat{Y}_{i,j})|^q]= 4^q\E[\sup_{r\in R}|\sum_{\pair{i,j}\in E}w_{i,j}\tilde{h}_r(X_i,X_j^\prime,\tilde{Y}_{i,j})|^q]
    \end{align*}
    \SingleColumnEnd
    which concludes the proof of \eqref{eq:weighted_u_statistics_decoupling}.

    By the symmetry of $\tilde{h}_r$ in the sense that $\tilde{h}_r(X_i,X_j,Y_{i,j})=\tilde{h}_r(X_j,X_i,Y_{j,i})$, we have
    \DoubleColumn
    \begin{align*}
        \E[&\sup_{r\in R} |\sum_{\pair{i,j}\in E}w_{i,j}\tilde{h}_r(X_i,X_j^\prime)|^q]\\
        &=\E[\sup_{r\in R} |\frac{1}{2}\sum_{\pair{i,j}\in E}w_{i,j}(\tilde{h}_r(X_i,X_j^\prime,\tilde{Y}_{i,j})\\
        &\qquad+\tilde{h}_r(X_i^\prime,X_j,\tilde{Y}_{i,j}))|^q]\\
        &= \E[\sup_{r\in R} |\frac{1}{2}\sum_{\pair{i,j}\in E}w_{i,j}(\tilde{h}_r(X_i,X_j^\prime,\tilde{Y}_{i,j})\\
        &\qquad+\tilde{h}_r(X_i^\prime,X_j,\tilde{Y}_{i,j})+\tilde{h}_r(X_i,X_j,Y_{i,j})\\
        &\qquad+\tilde{h}_r(X_i^\prime,X_j^\prime,Y^\prime_{i,j})\\
        &\qquad-\frac{1}{2}\sum_{\pair{i,j}\in E}w_{i,j}\tilde{h}_r(X_i,X_j,Y_{i,j})\\
        &\qquad-\frac{1}{2}\sum_{\pair{i,j}\in E}w_{i,j}\tilde{h}_r(X_i^\prime,X_j^\prime,Y^\prime_{i,j})|^q]\\
        &(\text{Triangle's Inequality and the convexity of }\sup_{r\in R}|\cdot|^q)\\
        &\le \frac{1}{2}\E[\sup_{r\in R} |\sum_{\pair{i,j}\in E}w_{i,j}(\tilde{h}_r(X_i,X_j^\prime,\tilde{Y}_{i,j})\\
        &\qquad+\tilde{h}_r(X_i^\prime,X_j,\tilde{Y}_{i,j})+\tilde{h}_r(X_i,X_j,Y_{i,j})\\
        &\qquad+\tilde{h}_r(X_i^\prime,X_j^\prime,Y^\prime_{i,j})|^q]\\
        &\qquad+ \frac{1}{4}\E[\sup_{r\in R} |2\sum_{\pair{i,j}\in E}w_{i,j}\tilde{h}_r(X_i,X_j,Y_{i,j})|^q] \\
        &\qquad+ \frac{1}{4}\E[\sup_{r\in R} |2\sum_{\pair{i,j}\in E}w_{i,j}\tilde{h}_r(X_i^\prime,X_j^\prime,Y^\prime_{i,j})|^q]\\
        &= \frac{1}{2}\E[\sup_{r\in R} |4\sum_{\pair{i,j}\in E}w_{i,j}\E_\sigma[\tilde{h}_r(Z_i,Z_j,\tilde{Y}_{i,j})]|^q]\\
        &\qquad+ \frac{1}{2}\E[\sup_{r\in R} |2\sum_{\pair{i,j}\in E}w_{i,j}\tilde{h}_r(X_i,X_j,Y_{i,j})|^q]\\
        &(\text{Jansen's Inequality and the convexity of }\sup_{r\in R}|\cdot|^q)\\
        &\le \frac{1}{2}\E[\sup_{r\in R} |4\sum_{\pair{i,j}\in E}w_{i,j}\tilde{h}_r(Z_i,Z_j,\tilde{Y}_{i,j})|^q] \\
        &\qquad+ \frac{1}{2}\E[\sup_{r\in R} |2\sum_{\pair{i,j}\in E}w_{i,j}\tilde{h}_r(X_i,X_j,Y_{i,j})|^q]\\
        &(\text{According to }\eqref{eq:weighted_u_statistics_decoupling_fact2})\\
        &= \frac{1}{2}\E[\sup_{r\in R} |4\sum_{\pair{i,j}\in E}w_{i,j}\tilde{h}_r(X_i,X_j,Y_{i,j})|^q] \\
        &\qquad+ \frac{1}{2}\E[\sup_{r\in R} |2\sum_{\pair{i,j}\in E}w_{i,j}\tilde{h}_r(X_i,X_j,Y_{i,j})|^q]\\
        &\le \E[\sup_{r\in R} |4\sum_{\pair{i,j}\in E}w_{i,j}\tilde{h}_r(X_i,X_j,Y_{i,j})|^q]\\
        &= 4^q\E[\sup_{r\in R} |\sum_{\pair{i,j}\in E}w_{i,j}\tilde{h}_r(X_i,X_j,Y_{i,j})|^q]
    \end{align*}
    \DoubleColumnEnd
    \SingleColumn
    \begin{align*}
        \E[&\sup_{r\in R} |\sum_{\pair{i,j}\in E}w_{i,j}\tilde{h}_r(X_i,X_j^\prime)|^q]\\
        &=\E[\sup_{r\in R} |\frac{1}{2}\sum_{\pair{i,j}\in E}w_{i,j}(\tilde{h}_r(X_i,X_j^\prime,\tilde{Y}_{i,j})+\tilde{h}_r(X_i^\prime,X_j,\tilde{Y}_{i,j}))|^q]\\
        &= \E[\sup_{r\in R} |\frac{1}{2}\sum_{\pair{i,j}\in E}w_{i,j}(\tilde{h}_r(X_i,X_j^\prime,\tilde{Y}_{i,j})+\tilde{h}_r(X_i^\prime,X_j,\tilde{Y}_{i,j})+\tilde{h}_r(X_i,X_j,Y_{i,j})+\tilde{h}_r(X_i^\prime,X_j^\prime,Y^\prime_{i,j})\\
        &\qquad-\frac{1}{2}\sum_{\pair{i,j}\in E}w_{i,j}\tilde{h}_r(X_i,X_j,Y_{i,j})-\frac{1}{2}\sum_{\pair{i,j}\in E}w_{i,j}\tilde{h}_r(X_i^\prime,X_j^\prime,Y^\prime_{i,j})|^q]\\
        &(\text{Triangle's Inequality and the convexity of }\sup_{r\in R}|\cdot|^q)\\
        &\le \frac{1}{2}\E[\sup_{r\in R} |\sum_{\pair{i,j}\in E}w_{i,j}(\tilde{h}_r(X_i,X_j^\prime,\tilde{Y}_{i,j})+\tilde{h}_r(X_i^\prime,X_j,\tilde{Y}_{i,j})+\tilde{h}_r(X_i,X_j,Y_{i,j})+\tilde{h}_r(X_i^\prime,X_j^\prime,Y^\prime_{i,j})|^q]\\
        &\qquad+\frac{1}{4}\E[\sup_{r\in R} |2\sum_{\pair{i,j}\in E}w_{i,j}\tilde{h}_r(X_i,X_j,Y_{i,j})|^q] + \frac{1}{4}\E[\sup_{r\in R} |2\sum_{\pair{i,j}\in E}w_{i,j}\tilde{h}_r(X_i^\prime,X_j^\prime,Y^\prime_{i,j})|^q]\\
        &= \frac{1}{2}\E[\sup_{r\in R} |4\sum_{\pair{i,j}\in E}w_{i,j}\E_\sigma[\tilde{h}_r(Z_i,Z_j,\tilde{Y}_{i,j})]|^q]+ \frac{1}{2}\E[\sup_{r\in R} |2\sum_{\pair{i,j}\in E}w_{i,j}\tilde{h}_r(X_i,X_j,Y_{i,j})|^q]\\
        &(\text{Jansen's Inequality and the convexity of }\sup_{r\in R}|\cdot|^q)\\
        &\le \frac{1}{2}\E[\sup_{r\in R} |4\sum_{\pair{i,j}\in E}w_{i,j}\tilde{h}_r(Z_i,Z_j,\tilde{Y}_{i,j})|^q] + \frac{1}{2}\E[\sup_{r\in R} |2\sum_{\pair{i,j}\in E}w_{i,j}\tilde{h}_r(X_i,X_j,Y_{i,j})|^q]\\
        &(\text{According to }\eqref{eq:weighted_u_statistics_decoupling_fact2})\\
        &= \frac{1}{2}\E[\sup_{r\in R} |4\sum_{\pair{i,j}\in E}w_{i,j}\tilde{h}_r(X_i,X_j,Y_{i,j})|^q] + \frac{1}{2}\E[\sup_{r\in R} |2\sum_{\pair{i,j}\in E}w_{i,j}\tilde{h}_r(X_i,X_j,Y_{i,j})|^q]\\
        &\le \E[\sup_{r\in R} |4\sum_{\pair{i,j}\in E}w_{i,j}\tilde{h}_r(X_i,X_j,Y_{i,j})|^q]\\
        &= 4^q\E[\sup_{r\in R} |\sum_{\pair{i,j}\in E}w_{i,j}\tilde{h}_r(X_i,X_j,Y_{i,j})|^q]
    \end{align*}
    \SingleColumnEnd
    which \eqref{eq:weighted_u_statistic_coupling} follows.

\subsection{Proof of Lemma \ref{le:weighted_randomization}} % (fold)
\label{sub:proof_of_lemma_le:weighted_randomization}

    Re-using the notations used in the proof of Lemma \ref{le:weighted_u_statistics_decoupling}, we further introduce $(X_i^{\prime\prime})_{i=1}^n$, a copy of $(X_i^\prime)_{i=1}^n$, independent from $\Sigma$, $\Sigma^\prime$, and denote by $\Sigma^{\prime\prime}$ its sigma-field. Let $(\tilde{Y}_{i,j}^{\prime\prime})_{(i,j)\in E}$ Bernoulli random variables such that $\Pro[\tilde{Y}_{i,j}^{\prime\prime}=1\mid \Sigma,\Sigma^\prime,\Sigma^{\prime\prime}]=\eta(X_i,X_j^{\prime\prime})$. We now use classic randomization techniques and introduce our "ghost" sample:
    \DoubleColumn
\begin{align*}
    &\E[\sup_{r\in R}|\sum_{\pair{i,j}\in E} w_{i,j}\tilde{h}_r(X_i,X_j^\prime,\tilde{Y}_{i,j}^{\prime\prime})|^q]\\
    &(\tilde{h}_r \text{ is degenerated})\\
    &= \E[\sup_{r\in R}|\sum_{\pair{i,j}\in E} w_{i,j}(\tilde{h}_r(X_i,X_j^\prime,\tilde{Y}_{i,j}^{\prime\prime})\\
    &\qquad-\E_{\Sigma^{\prime\prime}}[\tilde{h}_r(X_i,X_j^{\prime\prime},\tilde{Y}_{i,j}^{\prime\prime})])|^q]\\
    &(\text{Jansen's Inequality})\\
    &\le \E[\sup_{r\in R}|\sum_{\pair{i,j}\in E} w_{i,j}(\tilde{h}_r(X_i,X_j^\prime,\tilde{Y}_{i,j}^{\prime\prime})\\
    &\qquad-\tilde{h}_r(X_i,X_j^{\prime\prime},\tilde{Y}_{i,j}^{\prime\prime}))|^q]\\
    &= \E[\sup_{r\in R}|\sum_{j=1}^n \sum_{i:\pair{i,j}\in E} w_{i,j}(\tilde{h}_r(X_i,X_j^\prime,\tilde{Y}_{i,j}^{\prime\prime})\\
    &\qquad-\tilde{h}_r(X_i,X_j^{\prime\prime},\tilde{Y}_{i,j}^{\prime\prime}))|^q]
\end{align*}
\DoubleColumnEnd
    \SingleColumn
\begin{align*}
    \E[&\sup_{r\in R}|\sum_{\pair{i,j}\in E} w_{i,j}\tilde{h}_r(X_i,X_j^\prime,\tilde{Y}_{i,j}^{\prime\prime})|^q]\\
    &(\tilde{h}_r \text{ is degenerated})\\
    &= \E[\sup_{r\in R}|\sum_{\pair{i,j}\in E} w_{i,j}(\tilde{h}_r(X_i,X_j^\prime,\tilde{Y}_{i,j}^{\prime\prime})-\E_{\Sigma^{\prime\prime}}[\tilde{h}_r(X_i,X_j^{\prime\prime},\tilde{Y}_{i,j}^{\prime\prime})])|^q]\\
    &(\text{Jansen's Inequality})\\
    &\le \E[\sup_{r\in R}|\sum_{\pair{i,j}\in E} w_{i,j}(\tilde{h}_r(X_i,X_j^\prime,\tilde{Y}_{i,j}^{\prime\prime})-\tilde{h}_r(X_i,X_j^{\prime\prime},\tilde{Y}_{i,j}^{\prime\prime}))|^q]\\
    &= \E[\sup_{r\in R}|\sum_{j=1}^n \sum_{i:\pair{i,j}\in E} w_{i,j}(\tilde{h}_r(X_i,X_j^\prime,\tilde{Y}_{i,j}^{\prime\prime})-\tilde{h}_r(X_i,X_j^{\prime\prime},\tilde{Y}_{i,j}^{\prime\prime}))|^q]
\end{align*}
\SingleColumnEnd
Let $(\sigma_i)_{i=1}^n$ be independent Rademacher variables, independent of $\Sigma$, $\Sigma^\prime$ and $\Sigma^{\prime\prime}$, then we have:
\DoubleColumn
\begin{align*}
    \E[&\sup_{r\in R}|\sum_{j=1}^n \sum_{i:\pair{i,j}\in E} w_{i,j}(\tilde{h}_r(X_i,X_j^\prime,\tilde{Y}_{i,j})\\
    &\qquad-\tilde{h}_r(X_i,X_j^{\prime\prime},\tilde{Y}_{i,j}^{\prime\prime}))|^q\mid \Sigma]\\
    &= \E[\sup_{r\in R}|\sum_{j=1}^n \sigma_j \sum_{i:\pair{i,j}\in E} w_{i,j}(\tilde{h}_r(X_i,X_j^\prime,\tilde{Y}_{i,j})\\
    &\qquad-\tilde{h}_r(X_i,X_j^{\prime\prime},\tilde{Y}_{i,j}^{\prime\prime}))|^q\mid \Sigma]\\
    &(\text{Triangle's Inequality and the convexity of }\sup_{r\in R}|\cdot|^q)\\
    &\le 2^q\E[\sup_{r\in R}|\sum_{j=1}^n \sigma_j \sum_{i:\pair{i,j}\in E} w_{i,j}\tilde{h}_r(X_i,X_j^\prime,\tilde{Y}_{i,j})|^q\mid \Sigma]\\
    &+ 2^q\E[\sup_{r\in R}|\sum_{j=1}^n \sigma_j \sum_{i:\pair{i,j}\in E} w_{i,j}\tilde{h}_r(X_i,X_j^{\prime\prime},\tilde{Y}_{i,j}^{\prime\prime}))|^q\mid \Sigma]\\
    &\le 2^q\E[\sup_{r\in R}|\sum_{j=1}^n\sigma_j\sum_{i:\pair{i,j}\in E}w_{i,j} \tilde{h}_r(X_i,X_j^\prime,\tilde{Y}_{i,j})|^q\mid \Sigma]
\end{align*}
\DoubleColumnEnd
\SingleColumn
\begin{align*}
    \E[&\sup_{r\in R}|\sum_{j=1}^n \sum_{i:\pair{i,j}\in E} w_{i,j}(\tilde{h}_r(X_i,X_j^\prime,\tilde{Y}_{i,j})-\tilde{h}_r(X_i,X_j^{\prime\prime},\tilde{Y}_{i,j}^{\prime\prime}))|^q\mid \Sigma]\\
    &= \E[\sup_{r\in R}|\sum_{j=1}^n \sigma_j \sum_{i:\pair{i,j}\in E} w_{i,j}(\tilde{h}_r(X_i,X_j^\prime,\tilde{Y}_{i,j})-\tilde{h}_r(X_i,X_j^{\prime\prime},\tilde{Y}_{i,j}^{\prime\prime}))|^q\mid \Sigma]\\
    &(\text{Triangle's Inequality and the convexity of }\sup_{r\in R}|\cdot|^q)\\
    &\le 2^q\E[\sup_{r\in R}|\sum_{j=1}^n \sigma_j \sum_{i:\pair{i,j}\in E} w_{i,j}\tilde{h}_r(X_i,X_j^\prime,\tilde{Y}_{i,j})|^q\mid \Sigma]\\
    &\qquad+ 2^q\E[\sup_{r\in R}|\sum_{j=1}^n \sigma_j \sum_{i:\pair{i,j}\in E} w_{i,j}\tilde{h}_r(X_i,X_j^{\prime\prime},\tilde{Y}_{i,j}^{\prime\prime}))|^q\mid \Sigma]\\
    &\le 2^q\E[\sup_{r\in R}|\sum_{j=1}^n\sigma_j\sum_{i:\pair{i,j}\in E}w_{i,j} \tilde{h}_r(X_i,X_j^\prime,\tilde{Y}_{i,j})|^q\mid \Sigma]
\end{align*}
\SingleColumnEnd
and get
\DoubleColumn
\begin{align*}
    \E[\sup_{r\in R}&|\sum_{\pair{i,j}\in E} w_{i,j}\tilde{h}_r(X_i,X_j^\prime,\tilde{Y}_{i,j})|^q] \\
    &\le 2^q\E[\sup_{r\in R}|\sum_{j=1}^n\sigma_j\sum_{i:\pair{i,j}\in E}w_{i,j} \tilde{h}_r(X_i,X_j^\prime,\tilde{Y}_{i,j})|^q]
\end{align*}
\DoubleColumnEnd
\SingleColumn
\begin{align*}
    \E[\sup_{r\in R}|\sum_{\pair{i,j}\in E} w_{i,j}\tilde{h}_r(X_i,X_j^\prime,\tilde{Y}_{i,j})|^q] \le 2^q\E[\sup_{r\in R}|\sum_{j=1}^n\sigma_j\sum_{i:\pair{i,j}\in E}w_{i,j} \tilde{h}_r(X_i,X_j^\prime,\tilde{Y}_{i,j})|^q]
\end{align*}
\SingleColumnEnd
Then repeating the same argument but for the $(X_i)_{i=1}^n$ will give the similar inequality. The desired inequality will follows putting these two inequalities togather.

% section proofs_of_the_appendix (end)
\subsection{Proof of Theorem~\ref{th:moment_inequality_of_weighted_u_statistics}} % (fold)
\label{sub:proof_of_theorem_}

By the decoupling, undecoupling and randomization techniques (see Lemma~\ref{le:weighted_u_statistics_decoupling}, Lemma~\ref{le:weighted_u_statistics_randomization}), the symmetry and the degeneration of $f$ and the symmetry of $(w_{i,j})_{\pair{i,j}\in E}$, we have
\begin{align*}
    \E[\sup_f&|\sum_{\pair{i,j}\in E}w_{i,j} f(X_i,X_j)|^q]\\
    &\le 16^q\E[\sup_f|\sum_{\pair{i,j}\in E}w_{i,j}\rademacher_i\rademacher_j^\prime f(X_i, X_j^\prime)|^q]\\
    &\le 64^q\E[\sup_f|\sum_{\pair{i,j}\in E}w_{i,j}\rademacher_i\rademacher_j f(X_i, X_j)|^q]
\end{align*}
It means we can convert the moment of the original $U$-process to the moment of Rademacher chaos which can be handled by moment inequalities of \cite{Boucheron2005}.

In particular, for any $q\ge 2$,
\begin{align*}
    (\E_\rademacher[Z_\rademacher^q])^{1/q}
    % &=(\E_\rademacher[(Z_\rademacher-\E_\rademacher[Z_\rademacher]+\E_\rademacher[Z_\rademacher])^q])^{1/q}\\
    % &(\text{Triangle's Inequality, the convexity of }|\cdot|^q\text{ and }Z_\rademacher\ge0)\\
    &\le \E_\rademacher[Z_\rademacher] + (E_\rademacher[(Z_\rademacher-\E_\rademacher[Z_\rademacher])_+^q])^{1/q}\\
    &\le \E_\rademacher[Z_\rademacher]+3\sqrt{q}\E_\rademacher U_\rademacher + 4qB
\end{align*}
where $B$ is defined below
\[B=\sup_f\sup_{\alpha,\alpha^\prime:\|\alpha\|_2,\|\alpha^\prime\|_2\le 1}|\sum_{\pair{i,j}\in E}w_{i,j}\alpha_i\alpha_j^\prime f(X_i,X_j)|.\]
The second inequality above follows by Theorem~14 of \cite{Boucheron2005}.

Using the inequality $(a+b+c)^q\le 3^{(q-1)}(a^q+b^q+c^q)$ valid for $q\ge 2,a,b,c>0$, we have
\[\E_\rademacher[Z_\rademacher^q]\le 3^{q-1}(\E_\rademacher[Z_\rademacher]^q+3^qq^{q/2}\E_\rademacher[U_\rademacher]^q+4^qq^qB^q).\]
It remains to derive suitable upper bounds for the expectation of the three terms on the right hand side.

\textit{First term:} $\E[\E_\rademacher[Z_\rademacher]^q]$. Using the symmetrization trick, we have
\[\E[\E_\rademacher[Z_\rademacher]^q]\le 4^q\E[\E_\rademacher[Z_\rademacher^\prime]^q]\]
which $Z_\rademacher^\prime = \sup_f|\sum_{\pair{i,j}\in E}\rademacher_i\rademacher_j^\prime f(X_i,X_j)|$. Note that $\E_\rademacher$ now denotes expectation taken with respect to both the $\rademacher$ and the $\rademacher^\prime$. For simplicity, we denote by $A=\E_{\rademacher}[Z_\rademacher^\prime]$. In order to apply Corollary~3 of \cite{Boucheron2005}, define, for $k=1,\dots,n$, the random variables
\[A_k = \E_{\rademacher}[\sup_f |\sum_{\pair{i,j}\in E,i,j\neq k}w_{i,j}\rademacher_i\rademacher_j^\prime f(X_i,X_j)|].\]
It is easy to see that $A_k\le A$.

On the other hand, defining
\[R_k=\sup_f|\sum_{i:\pair{i,k}\in E}w_{i,k}\rademacher_if(X_i,X_k)|,\]
\[M=\max_kR_k\]
and denoting by $f^*$ the function achieving the maximum in the definition of $Z$, we clearly have
\begin{align*}
    A-A_k
    % &\le \E_\rademacher[\sum_{j:\pair{k,j}\in E}w_{k,j}\rademacher_k\rademacher_j^\prime f^*(X_k,X_j)\\
    % &\qquad+\sum_{i:\pair{i,k}\in E}w_{i,k}\rademacher_i\rademacher_k^\prime f^*(X_i,X_k)]\\
    &\le 2\E_\rademacher[M]
\end{align*}
and
\begin{align*}
    \sum_{k=1}^n(A-A_k)
    % &\le \E_\rademacher[\sum_{k=1}^n\rademacher_k\sum_{j:\pair{k,j}\in E}w_{k,j}\rademacher_j^\prime f^*(X_k,X_j)\\
    % &\qquad+\sum_{k=1}^n\rademacher_k^\prime\sum_{i:\pair{i,k}\in E}w_{i,k}\rademacher_i f^*(X_i,X_k)]\\
    &\le 2A.
\end{align*}
Therefore,
\[\sum_{k=1}^n(A-A_k)^2\le 4A\E_\rademacher M.\]
Then by Corollary~3 of \cite{Boucheron2005}, we obtain
\begin{equation}
    \label{eq:inequality_rademacher_chaos_1}
    \E[\E_\rademacher[Z_\rademacher^\prime]^q]\le 2^{q-1}(2^q(E[Z_\rademacher^\prime])^q+5^qq^q\E[\E_\rademacher[M]^q])
\end{equation}

To bound $\E[\E_\rademacher[M]^q]$, observe that $\E_\rademacher[M]$ is a conditional Rademacher average, for which Theorem~13 of \cite{Boucheron2005} could be applied. Since $\max_k\sup_{f,i}w_{i,k}f(X_i,X_k)\le \|\weight{}\|_\infty F$, we have
\begin{equation}
    \label{eq:inequality_rademacher_chaos_2}
    \E[\E_\rademacher[M]^q]\le 2^{q-1}(2^q\E[M]^q+5^qq^q\|\weight{}\|_\infty^qF^q).
\end{equation}
By undecoupling, we have $\E[Z_\rademacher^\prime]=\E[\E_{\rademacher,\rademacher^\prime}[Z_\rademacher^\prime]]\le\E[4\E_\rademacher[Z_\rademacher]]=4\E[Z_\rademacher]$. Collecting all terms, we have
\DoubleColumn
\begin{align*}
    \E[\E_\rademacher[Z_\rademacher]^q]&\le 64^q\E[Z_\rademacher]^q+160^qq^q\E[M]^q\\
    &+400^q\|\weight{}\|_\infty^qF^qq^{2q}.
\end{align*}
\DoubleColumnEnd
\SingleColumn
\begin{align*}
    \E[\E_\rademacher[Z_\rademacher]^q]\le 64^q\E[Z_\rademacher]^q+160^qq^q\E[M]^q+400^q\|\weight{}\|_\infty^qF^qq^{2q}.
\end{align*}
\SingleColumnEnd

\textit{Second term:} $\E[\E_\rademacher[U_\rademacher]^q].$

By the Cauchy-Schwarz inequality we can observe that
\[\sup_{f,i}\sup_{\alpha:\|\alpha\|_2\le 1}\sum_{j:\pair{i,j}\in E}w_{i,j}\alpha_jf(X_i,X_j)\le \|\weight{}\|_{\max}F.\]
Then similar to the bound of $\E[\E_\rademacher[M]^q]$, we have
\[\E[\E_\rademacher[U_\rademacher]^q]\le 2^{q-1}(2^q\E[U_\rademacher]^q+5^qq^q\|\weight{}\|_{\max}^qF^q).\]

\textit{Third term:} $\E[B^q]$. By the Cauchy-Schwarz inequality, we have $B\le \sqrt{\sum_{\pair{i,j}\in E}w_{i,j}^2}F= \|\weight{}\|_2F$ so
\[\E[B^q]\le \|\weight{}\|_2^qF^q.\]

Now it remains to simply put the pieces together to obtain
\DoubleColumn
\begin{align*}
\E[\sup_f&|\sum_{\pair{i,j}\in E}w_{i,j} f(X_i,X_j)|^q]\\
&\le C(\E[Z_\rademacher]^q+q^{q/2}\E[U_\rademacher]^q+q^q\E[M]^q\\
&\qquad+\|\weight{}\|_\infty^q F^qq^{2q}+\|\weight{}\|_{\max}^qF^qq^{3q/2}\\
&\qquad+F^q\|\weight{}\|_2^qq^q)
\end{align*}
\DoubleColumnEnd
\SingleColumn
\begin{align*}
\E[\sup_f&|\sum_{\pair{i,j}\in E}w_{i,j} f(X_i,X_j)|^q]\\
&\le C(\E[Z_\rademacher]^q+q^{q/2}\E[U_\rademacher]^q+q^q\E[M]^q+\|\weight{}\|_\infty^q F^qq^{2q}+\|\weight{}\|_{\max}^qF^qq^{3q/2}+F^q\|\weight{}\|_2^qq^q)
\end{align*}
\SingleColumnEnd
for an appropriate constant $C$. 
In order to derive the exponential inequality, we use Markov's inequality $\Pro[X\ge t]\le t^{-q}\E[Z^q]$ and choose
\DoubleColumn
\begin{align*}
    q&=C\min\Big((\frac{t}{\E[U_\rademacher]})^2,\frac{t}{\E[M]},\frac{t}{F\|\weight{}\|_2},(\frac{t}{\|\weight{}\|_{\max} F})^{2/3},\\
    &\qquad\sqrt{\frac{t}{\|\weight{}\|_\infty F}}\Big)
\end{align*}
\DoubleColumnEnd
\SingleColumn
\begin{align*}
    q=C\min\Big((\frac{t}{\E[U_\rademacher]})^2,\frac{t}{\E[M]},\frac{t}{F\|\weight{}\|_2},(\frac{t}{\|\weight{}\|_{\max} F})^{2/3},\sqrt{\frac{t}{\|\weight{}\|_\infty F}}\Big)
\end{align*}
\SingleColumnEnd
for an appropriate constant $C$.

% subsection proofs_of_main_track (end)

% \begin{proof}[Proof of Theorem 2]
%     Since the solution $(a^*,b^*,\probdistri{}^*)$ and the vector $\bmlambda^*$ satisfy the KKT condition of program (10), according to \eqref{eq:lagrangian_function}, we have
%     \[\partial L/\partial a = \frac{1}{1+r}a^{*-r/(1+r)}-\sum_{i=1}^n\lambda_i^{(3)*}n_i - \lambda^{(8)*} = 0\]
%     then
%     \[\frac{1}{1+r}a^{*-r/(1+r)}=\sum_{i=1}^n\lambda_i^{(3)*}n_i - \lambda^{(8)*}.\]
%     Let us consider a new program called (P) that we add a constrain $a\le a^*-\epsilon$ to (10). The KKT condition of (P) is the same as (10) expect from the contains of $a$. We write the lagrangian function of (P) as
%     \begin{equation}
%         \label{eq:lagrangian_function}
%         L^\prime(a,b,\probdistri{},\bmlambda)=L(a,b,\probdistri{},\bmlambda)+\lambda^{(10)}(a-a^*+\epsilon),
%     \end{equation}
%     then
%     \begin{align*}
%         &\partial L^\prime/\partial a = \frac{1}{1+r}(a^*-\epsilon)^{-r/(1+r)}-\sum_{i=1}^n\lambda_i^{(3)*}n_i - \lambda^{(8)*}\\
%         &\qquad - (\frac{1}{1+r}(a^*-\epsilon)^{-r/(1+r)} - \frac{1}{1+r}a^{*-r/(1+r)}) = 0
%     \end{align*}
%     and
%     \[\frac{1}{1+r}((a^*-\epsilon)^{-r/(1+r)} - a^{*-r/(1+r)})(a^*-\epsilon-a^*+\epsilon)=0.\]
%     So the solution $(a^*-\epsilon,b^*,\probdistri{}^*)$ and the vector $(\bmlambda^*,\lambda^{(10)*})$ where
%     \[\lambda^{(10)*}= \frac{1}{1+r}((a^*-\epsilon)^{-r/(1+r)} - a^{*-r/(1+r)})\]
%     also satisfy the KKT condition of (P).
    
%     Next, we will show that $(a^*-\epsilon,b^*,\probdistri{}^*)$ is optimal for (P) by the invex of the Lagrangian function of (P) and the sufficient optimality condition of \citep{Hanson1981On}.
%     Since $\bmlambda^*\ge 0$, the parts of $\probdistri{}$ in \eqref{eq:lagrangian_function} is convex (the only negative part is linear) and the parts of $b$ in \eqref{eq:lagrangian_function} is linear. When $a\le a^*$, the part of $a$
%     \begin{align*}
%         a^{1/(1+r)}    &- \sum_{i=1}^n\lambda_i^{(3)*}n_ia-\lambda^{(8)*}a\\
%         &=a^{1/(1+r)} - a(\frac{1}{1+r}(a^*-\epsilon)^{-r/(1+r)}-\lambda^{(10)*})\\
%         &= a^{1/(1+r)} - \frac{a}{1+r}a^{*-r/(1+r)}
%     \end{align*}
%     is monotonous.
%     Choosing the vector function $\eta=(\eta_a,\eta_b,(\eta_{i,j})_{\pair{i,j}\in E})$ such that
%     \begin{align*}
%         &\eta_a(a_1,a_2)\\
%         &= \frac{(1+r)(a_1^{1/(1+r)}-a_2^{1/(1+r)})+(a_2-a_1)a^{*-r/(1+r)}}{a_2^{-r/(1+r)}-a^{*-r/(1+r)}},
%     \end{align*}
%     $\eta_b(b_1,b_2)=(b_1-b_2)$ and $\eta_{i,j}(p^{(1)}_{i,j},p^{(2)}_{i,j})=p^{(1)}_{i,j}-p^{(2)}_{i,j}$ for all $\pair{i,j}\in E$, we have
%     \begin{align*}
%         L^\prime(&a_1,b_1,\probdistri{}_1,\bmlambda^*)-L^\prime(a_2,b_2,\probdistri{}_2,\bmlambda^*)\\
%         &\ge \eta((a_1,b_1,\probdistri{}_1), (a_2,b_2,\probdistri{}_2))^{\rm T}\nabla L^\prime(a_2,b_2,\probdistri{}_2,\bmlambda^*)
%     \end{align*}
%     where $(a_1,b_1,\probdistri{}_1)$ and $(a_2,b_2,\probdistri{}_2)$ are feasible solutions of (P).
%     So $L^\prime(a_1,b_1,\probdistri{}_1,\bmlambda^*)$ is invex. By the sufficient optimality conditions introduced by \citep{Hanson1981On}, $(a^*-\epsilon,b^*,\probdistri{}^*)$ is optimal for (P).

%     Similarly, we add a constrain $a\ge a^*+\epsilon$ to (10) and have that $(a^*,b^*,\probdistri{}^*)$ is also optimal for this program.
%     Finally, we make a conclusion that if $(a^*,b^*,\probdistri{}^*)$ satisfies the KKT condition of (10), then $(a^*,b^*,\probdistri{}^*)$ is optimal for (10).
% \end{proof}
\subsection{Proof of Corollary~\ref{cor:covering_number_inequality}} % (fold)
\label{sub:proof_of_corollary_}

    From Theorem~3, it is easy to know
    \DoubleColumn
    \begin{equation}
        \begin{aligned}
            \label{eq:moment_inequality_of_weighted_u_statistics}
            \kappa=&C(\E[Z_\rademacher]+\max(\E[U_\rademacher]\sqrt{\log(1/\delta)},\E[M]\log(1/\delta),\\
            &\quad\log(1/\delta)\|\weight{}\|_2,(\log(1/\delta))^{3/2}\|\weight{}\|_{\max},\\
            &\quad (\log(1/\delta))^2\|\weight{}\|_\infty)).
        \end{aligned}
    \end{equation}
    \DoubleColumnEnd
    \SingleColumn
    \begin{equation}
        \begin{aligned}
            \label{eq:moment_inequality_of_weighted_u_statistics}
            \kappa=C(\E[Z_\rademacher]+\max(&\E[U_\rademacher]\sqrt{\log(1/\delta)},\E[M]\log(1/\delta),\log(1/\delta)\|\weight{}\|_2,(\log(1/\delta))^{3/2}\|\weight{}\|_{\max},\\
            &\quad (\log(1/\delta))^2\|\weight{}\|_\infty)).
        \end{aligned}
    \end{equation}
    \SingleColumnEnd
An important character of these Rademacher processes in \eqref{eq:moment_inequality_of_weighted_u_statistics} is that they all satisfy the Khinchine inequality \eqref{eq:kinchine_type_inequality_process}. For simplicity, we denote the weighted Rademacher processes of $Z_\rademacher$ by 
\[\{z_\rademacher(f)=\sum_{\pair{i,j}\in E_<}w_{i,j}\rademacher_i\rademacher_j\linebreak f(X_i,X_j), f\in\mathcal{F}\}.\]
where $E_<=\{(i,j): \{i,j\}\in E, i<j\}$. Let $z_\rademacher^\prime=z_\rademacher/\|\weight{}\|_2$, following Theorem \ref{th:kinchine_type_ineqaulity}, we can easily have that $z_\rademacher^\prime$ satisfies \eqref{eq:kinchine_type_inequality_process} with degree $2$. Thus from Theorem \ref{th:rademacher_process_chaining}, we have
\begin{equation}
    \label{eq:weighted_rademacher_process_prove_1}
    \E_\rademacher[\sup_{f,g}|z^\prime_\rademacher(f)-z^\prime_\rademacher(s)|]\le K\int_0^D\log N(\mathcal F,d,\epsilon)d\epsilon.
\end{equation}
where 
\DoubleColumn
\begin{align*}
    d(f,g)&=(\E_\rademacher[|z^\prime_\rademacher(f)-z^\prime_\rademacher(g)|^2])^{1/2}\\
    &=\frac{1}{\|\weight{}\|_2}(\E_\rademacher[|\sum_{\pair{i,j}\in E}w_{i,j}\rademacher_i\rademacher_j(f(X_i,X_j)\\
    &\qquad-g(X_i,X_j))|^2])^{1/2}.
\end{align*}
\DoubleColumnEnd
\SingleColumn
\begin{align*}
    d(f,g)&=(\E_\rademacher[|z^\prime_\rademacher(f)-z^\prime_\rademacher(g)|^2])^{1/2}\\
    &=\frac{1}{\|\weight{}\|_2}(\E_\rademacher[|\sum_{\pair{i,j}\in E}w_{i,j}\rademacher_i\rademacher_j(f(X_i,X_j)-g(X_i,X_j))|^2])^{1/2}.
\end{align*}
\SingleColumnEnd
Recall that $\sup_f \|f\|_\infty = F$, we have $D= 2F$.
Since this metirc function $d$ is intractable, we need to convert it to the $\lebesgue_\infty$ metric. For all $f,s$, we have
\DoubleColumn
\begin{align*}
    d(f,g)&=\frac{1}{\|\weight{}\|_2}(\E_\rademacher[|\sum_{\pair{i,j}\in E}w_{i,j}\rademacher_i\rademacher_j(f(X_i,X_j)\\
    &\qquad-g(X_i,X_j))|^2])^{1/2}\\
    &\le \lebesgue_\infty(f,g)
\end{align*}
\DoubleColumnEnd
\SingleColumn
\begin{align*}
    d(f,g)&=\frac{1}{\|\weight{}\|_2}(\E_\rademacher[|\sum_{\pair{i,j}\in E}w_{i,j}\rademacher_i\rademacher_j(f(X_i,X_j)-g(X_i,X_j))|^2])^{1/2}\\
    &\le \lebesgue_\infty(f,g)
\end{align*}
\SingleColumnEnd
The fact that if $\forall f,s\in\mathcal F, d_1(f,s)\le d_2(f,s)$ then $N(\mathcal F,d_1,\epsilon)\le N(\mathcal F, d_2,\epsilon)$ combined with \eqref{eq:weighted_rademacher_process_prove_1} will give
\begin{align*}
    \E[Z_\rademacher]&= 2\E[\E_\rademacher[\sup_f|z_\rademacher(f)|]]\\
    &\le 2K\|\weight{}\|_2\int_0^{2F}\log N(\mathcal F,d,\epsilon)d\epsilon\\
    &\le 2K\|\weight{}\|_2\int_0^{2F}\log N_\infty(\mathcal F, \epsilon)d\epsilon
\end{align*}
where $K$ is a universal constant. Similarly, we can also bound $\E[M]$ by $K\|\weight{}\|_{\max}\linebreak\int_0^{2F}\sqrt{\log N_\infty(\mathcal F, \epsilon)}d\epsilon$. 
For $U_\rademacher$, let $\alpha^*$ be the (random) vector that maximizes $U_\rademacher$ and define
\[\{u_\rademacher(f) = \sum_{i=1}^n\rademacher_i\sum_{\pair{i,j}\in\E}w_{i,j}\alpha_j^*\linebreak f(X_i,X_j),f\in\mathcal{F}\}.\]
Clearly, $u_\rademacher$ satisfies the Kinchine inequality with degree $1$. Also, we need to convert its metric distance, 
\DoubleColumn
\begin{align*}
    (\E_\rademacher&[|u_\rademacher(f)-u_\rademacher(g)|^2])^{1/2}\\
    &=(\E_\rademacher[|\sum_{\pair{i,j}\in\E}w_{i,j}\rademacher_i\alpha_j^*(f(X_i,X_j)\\
    &\qquad-g(X_i,X_j))|^2])^{1/2}\\
    % &\le \sqrt{\sum_{\pair{i,j}\in\E}(w_{i,j}\alpha_j^*)^2}\sup_{X_i,X_j\in\mathcal X^2}|f(X_i,X_j)\\
    % &\qquad-g(X_i,X_j)|\\
    &\le \|\weight{}\|_2\lebesgue_\infty(f,g)
\end{align*}
\DoubleColumnEnd 
\SingleColumn
\begin{align*}
    (\E_\rademacher[|u_\rademacher(f)-u_\rademacher(g)|^2])^{1/2}&=(\E_\rademacher[|\sum_{\pair{i,j}\in\E}w_{i,j}\rademacher_i\alpha_j^*(f(X_i,X_j)-g(X_i,X_j))|^2])^{1/2}\\
    % &\le \sqrt{\sum_{\pair{i,j}\in\E}(w_{i,j}\alpha_j^*)^2}\sup_{X_i,X_j\in\mathcal X^2}|f(X_i,X_j)\\
    % &\qquad-g(X_i,X_j)|\\
    &\le \|\weight{}\|_2\lebesgue_\infty(f,g)
\end{align*}
\SingleColumnEnd
Thus, $\E[U_\rademacher]\le K\|\weight{}\|_2\int_0^{2F}\sqrt{\log N_\infty(\mathcal F,\epsilon)}d\epsilon$.

Pluging all these part into \eqref{eq:moment_inequality_of_weighted_u_statistics} will complete the corollary.
% section technical_proofs (end)
\section{Metric entropy inequality} % (fold)
\label{sub:metric_entropy_inequality}

The following theorems are more or less classical and well known. We present them here for the sake of completeness.

\begin{theorem}[Khinchine inequality for rademacher chaos, {\citet[Theorem 3.2.1]{de2012decoupling}}]
    \label{th:kinchine_type_ineqaulity}
    Let $F$ be a normed vector space and let $\{\rademacher{}_i\}_{i=1}^\infty$ be a Rademacher sequence. Denote by
    \DoubleColumn
    \begin{align*}
        X=x+\sum_{i=1}^n x_i\rademacher_i+\sum_{i_1<i_2\le n}\linebreak x_{i_1,i_2}\rademacher{}_{i_1}\rademacher{}_{i_2}+\dots\\
        +\sum_{i_1<\dots<i_d\le n}x_{i_1\dots i_d}\rademacher{}_{i_1}\dots \rademacher{}_{i_d}        
    \end{align*}
    \DoubleColumnEnd
    \SingleColumn
    \begin{align*}
        X=x+\sum_{i=1}^n x_i\rademacher_i+\sum_{i_1<i_2\le n}\linebreak x_{i_1,i_2}\rademacher{}_{i_1}\rademacher{}_{i_2}+\dots+\sum_{i_1<\dots<i_d\le n}x_{i_1\dots i_d}\rademacher{}_{i_1}\dots \rademacher{}_{i_d}        
    \end{align*}
    \SingleColumnEnd
    the rademacher chaos of order d. Let $1<p\le q<\infty$ and let
    \[\gamma=(\frac{p-1}{q-1})^{1/2}.\]
    Then, for all $d\ge 1$,
    \DoubleColumn
    \begin{align*}
        (\E[|x&+\sum_{i=1}^n \gamma x_i\rademacher_i+\sum_{i_1<i_2\le n}\gamma^2x_{i_1,i_2}\rademacher{}_{i_1}\rademacher{}_{i_2}+\dots\\
        &\qquad+\sum_{i_1<\dots<i_d\le n}\gamma^dx_{i_1\dots i_d}\rademacher{}_{i_1}\dots \rademacher{}_{i_d}|^q])^{1/q}\\
        &\le (\E[|x+\sum_{i=1}^n x_i\rademacher_i+\sum_{i_1<i_2\le n}x_{i_1,i_2}\rademacher{}_{i_1}\rademacher{}_{i_2}+\dots\\
        &\qquad+\sum_{i_1<\dots<i_d\le n}x_{i_1\dots i_d}\rademacher{}_{i_1}\dots \rademacher{}_{i_d}|^p])^{1/p}
    \end{align*}
    \DoubleColumnEnd
    \SingleColumn
    \begin{align*}
        (\E[|x&+\sum_{i=1}^n \gamma x_i\rademacher_i+\sum_{i_1<i_2\le n}\gamma^2x_{i_1,i_2}\rademacher{}_{i_1}\rademacher{}_{i_2}+\dots+\sum_{i_1<\dots<i_d\le n}\gamma^dx_{i_1\dots i_d}\rademacher{}_{i_1}\dots \rademacher{}_{i_d}|^q])^{1/q}\\
        &\le (\E[|x+\sum_{i=1}^n x_i\rademacher_i+\sum_{i_1<i_2\le n}x_{i_1,i_2}\rademacher{}_{i_1}\rademacher{}_{i_2}+\dots+\sum_{i_1<\dots<i_d\le n}x_{i_1\dots i_d}\rademacher{}_{i_1}\dots \rademacher{}_{i_d}|^p])^{1/p}
    \end{align*}
    \SingleColumnEnd
\end{theorem}

\begin{theorem}[metric entropy inequality, {\citet[Proposition 2.6]{Dembo1994}}]
    \label{th:rademacher_process_chaining}
    If a process $\{Y_f: f\in\mathcal F\}$ satisfies
\begin{equation}
    \label{eq:kinchine_type_inequality_process}
    (\E[|Y_f-Y_g|^p])^{1/p}\le (\frac{p-1}{q-1})^{m/2}(\E[|Y_f-Y_g|^q])^{1/q},
\end{equation}
for $1<q<p<\infty$ and some $m\ge 1$, and if
\begin{equation}
    \label{eq:process_distance}
    d(f,g) = (\E[|Y_f-Y_g|^2])^{1/2},
\end{equation}
there is a constant $K<\infty$ such that
    \begin{equation}
        \label{eq:rademacher_process_chaining}
        \E[\sup_{f,g}|Y_f-Y_g|]\le K\int_0^D(\log N(\mathcal F,d,\epsilon))^{m/2}d\epsilon.
    \end{equation}
    where $D$ is the $d$-diameter of $\mathcal{F}$.
\end{theorem}
\bibliography{Mendeley}
\bibliographystyle{apalike}

\end{document}